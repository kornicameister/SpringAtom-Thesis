\chapter{Podsumowanie}
\label{chapter:summary}
	Głównymi celami pracy było zaprojektowanie i przygotowanie aplikacji wspierającej warsztat samochodowy w zakresie
	zarządzania terminarzem wizyt, bazą klientów, informacjami o samym warsztacie oraz poznanie szkieletu aplikacji
	Spring, jako kompleksowego narzędzia wspierającego tworzenie rozbudowanych programów \textbf{Java EE}. 
	Znaczenie oraz korzyści jakie przynosi korzystanie z tego typu aplikacji nie sposób nie zauważyć. Dobrze zaprojektowana
	jest cennym dodatkiem wspomagającym pracę przedsiębiorstwa w zakresie zarówno chwili obecnej, jak i analizy danych historycznych.
	Program obejmujący swym zasięgiem globalny aspekt misji danej firmy, przy jednoczesnym dostępie do poszczególnych elementów. 
	
	W kontekście aplikacji demonstracyjnej nie udało się zrealizować wszystkich zamierzonych postulatów. Niemniej część z brakujących
	elementów, z uwagi na pracę oraz czas poświęcony na zaprojektowanie niewidocznych dla użytkownika elementów, co wcale nie umniejsza
	ich znaczenia, będzie możliwa do szybkiego wprowadzenia do gotowego rozwiązania. Pozostałe braki wynikają głównie z problematyki
	rozwiązania aspektów technicznych i dla celu aplikacji demonstracyjnej zostały pominięte.
	
	Udało się natomiast dobrze poznać podstawowe prawa i reguły rządzące pisaniem rozbudowanego projektu, zarówno w sensie ogólnym
	oraz przy użyciu szkieletu aplikacji Spring. Rzeczy, które mogą się wydawać trywialne, takie jak: prawidłowy dobór bibliotek, koncepcja przygotowania projektu przed jego implementacją,
	testy jednostkowe, a które zostały zaczerpnięte z zestawu \textbf{best practices} dla Spring, pozwoliły zrozumieć na co zwracać szczególną
	uwagę oraz jak ważne mogą stać się małe pomyłki. 