\chapter{Aplikacja demonstracyjna}
\label{chapter:app}

\section{Opis aplikacji wspomagającej warsztat samochodowy}
	
	Nadrzędnym celem aplikacji jest wsparcie dla misji przedsiębiorstwa\footnote{Misja przedsiębiorstwa - zestaw wartości opisujących rolę danego przedsiębiorstwa w jego otoczeniu} prowadzącego warsztat samochody, zajmujący się serwisowaniem samochodów, prowadzącym naprawy, przeglądy i dokonującym okresowych czynności eksploatacyjnych, jak na przykład wymiana oleju czy filtrów. Z tego powodu w kolejnych modułach aplikacji została zrealizowana część zarówno serwerowa, jak i kliencka dostarczająca funkcjonalności pozwalających na tworzenie, edycję oraz usuwanie obiektów biznesowych, a także przeglądanie informacji o nich. Duży nacisk został położony na zrealizowanie warstwy serwerowej, z uwagi na jej krytyczne znaczenie. Jest ona odpowiedzialna za realizację postulatów logiki biznesowej, zarządzanie prawami dostępu, walidację i konwersję danych. 
		
	\subsection{Funkcjonalność aplikacji}
	W części praktycznej zrealizowana została następująca funkcjonalność:
	\begin{enumerate}
		\item zarządzanie dostępem do poszczególnych stron, z wykorzystaniem uprawnień użytkowników:
		\begin{itemize}
			\item aplikacja rozpoznaje czy użytkownik jest zalogowany, dostosowując ilość dostępnych funkcji, w zależności od grupy (grup), w których użytkownik się znajduje,
			\item weryfikacja jest jednoetapowa,
			\item proces weryfikacji opiera się o adres internetowy,
		\end{itemize}
		\item zarządzanie spotkaniami:
		\begin{itemize}
			\item przeglądanie terminarza spotkań na poziomie dnia, tygodnia oraz miesiąca,
			\item tworzenie nowych spotkań, dostępne z poziomu terminarza,
		\end{itemize}
		\item tworzenie nowych użytkowników:
		\begin{itemize}
			\item nazwa użytkownika oraz hasło,
			\item uprawnienia,
			\item dane kontaktowe,
		\end{itemize}
		\item przeglądanie obiektów domenowych jako pojedynczych stron internetowych,
		\item przeglądanie danych w formie tabelarycznej,
		\item tworzenie nowych szablonów raportów biznesowych, zapisywanie ich oraz późniejsze generowanie wyniku
	\end{enumerate}

\section{Architektura MVC} 								\subsection{Model - Warstwa danych}
		Na warstwę modelu danych składają się klasy, zwane dalej obiektami domenowymi. Zdefiniowane w aplikacji praktycznej klasy, należące do tej warstwy, opisują obiekty rzeczywiste, związane ze specyfikacją działalności warsztatu samochodowego, a także takie, dzięki którym możliwe jest zarządzanie użytkownikami aplikacji oraz ich uprawnieniami. 
		
		W wybranych przypadkach (tabela \ref{app:model_list_of}), obiekty domenowe są wersjonowane. Oznacza to, że \textbf{Hibernate} trzyma historię zmian w bazie danych. Modyfikacja każdego lub jedynie wybranych pól (jest to zależne od konfiguracji danego obiektu domenowego), powoduje nie tyle zapisywanie zmian do bazy, co utworzenie nowych rekordów w tabelach:
	\begin{itemize}
		\item \textbf{revinfo} - zawiera następujące kolumny:
		\begin{itemize}
			\item datę utworzenia rewizji,
			\item login użytkownika, który dokonał zmiany,
		\end{itemize} 
		\item \textbf{revchanges} - zawiera następujące kolumny:
		\begin{itemize}
			\item numer rewizji,
			\item nazwę (pełna nazwa klasy) modelu,
		\end{itemize}
		\item \textbf{\{nazwa\_tabeli\}\_history} - zawiera ona te pola, które zostały wybrane, co powoduje utworzenie nowej rewizji obiektu domenowego.
	\end{itemize}
	
	Aby wskazać konkretny obiekt domenowy, dla którego wymagana jest wiedza o historii jego modyfikacji, należy użyć adnotacji \textbf{@Audited}. Jeśli zostanie nią opatrzona cała klasa, będzie to równoznaczne z tym, że wszystkie atrybuty tej klasy będą kandydatami do utworzenia nowej rewizji. Z drugiej strony możliwe jest podanie jedynie konkretnych pól. Fragment klasy \textbf{SCar} (listing \ref{app:scar_version}) pokazuje użycie adnotacji nad polem \textbf{licencePlate}. 
	\begin{code}
		\inputminted[
			linenos=true,
			firstline=61,
			lastline=65,
			fontfamily=monospace,
			obeytabs=true,
			samepage=true,
			fontsize=\scriptsize
		]{java}{../SpringAtom_thesis/src/main/java/org/agatom/springatom/server/model/beans/car/SCar.java}
		% src file used
		\caption[Użycie adnotacji \textbf{@Audited}]{
			Użycie adnotacji \textbf{@Audited}, źródło: opracowanie własne
		}
		\label{app:scar_version}
	\end{code}
	
	\begin{center}
		\begin{longtable}{| p{4cm} | p{2cm} | p{8cm} |}
			\caption[Lista obiektów domenowych]{Lista obiektów domenowych \label{app:model_list_of}}\tabularnewline	
			
			% header
			\hline
				\multicolumn{1}{|c|}{\textbf{Obiekt domenowy}} 		&
				\multicolumn{1}{|c|}{\textbf{Wersjonowany}} 			&
				\multicolumn{1}{|c|}{\textbf{Opis}} 					\\
			\hline
			\endfirsthead
			
			\multicolumn{2}{c}
			{{\bfseries \tablename\ \thetable{} -- kontynuacja...}} 	\\
			\hline
				\multicolumn{1}{|c|}{\textbf{Obiekt domenowy}} 		&
				\multicolumn{1}{|c|}{\textbf{Wersjonowany}} 			&
				\multicolumn{1}{|c|}{\textbf{Opis}} 					\\
			\hline
			\endhead
				
			\hline
				\multicolumn{3}{|r|}{{Następna strona...}} 			\\
			\hline
			\endfoot
	
			\hline
			\endlastfoot	
			% end of header
			
			% body starts her
			SAppointment 		& 
			Nie			 		&
			Pojedyncza wizyta danego pojazdu w warsztacie samochodowym, która wydarzyła się w konkretnym momencie czasu. Wizyta powiązana jest z konkretnym samochodem, będącym jej podmiotem oraz zawiera informacje 
			o osobie, która zarejestrowała zgłoszenie i była jego wykonawcą (mogą to być inne osoby), a także listę czynności, jakie należało wykonać.
			\hline
			SAppointmentTask 	& 
			Nie					&
			Pojedyncza czynność, wykonana podczas wizyty. Czynność opisana jest przez jej typ (może to być, na przykład, wymiana oleju) oraz komentarz (dla wymiany oleju może to być informacja o tym, jaki olej został wymieniony). 
			\hline
			SAppointmentIssue 	& 
			Nie					&
			Opisuje problemową sytuację, związaną z danym spotkaniem. Dla pojedynczej wizyty sytuacją wyjątkową może być fakt nieodbycia się spotkania, ponieważ klient warsztatu, do który należy pojazd (podmiot wizyty), nie stawił się
			na umówiony termin. Jednocześnie klasa \textbf{SAppointmentIssue} jest rozszerzeniem klasy \textbf{SIssue}, dziedziczy więc wszystkie jej atrybuty. 
			\hline
			SCarMaster 			&
			Nie					& 
			Zawiera informacje opisujące samochód, które nie zależą od jego konkretnej rewizji. Są to atrybuty takie jak marka, model, producent i kraj, z którego samochód pochodzi.
			\hline
			SCar 				&
			Tak 				& 
			Centralny obiekt domenowy systemu. Na jego opis składają się atrybuty zdefiniowane w klasie \textbf{SCarMaster}, a także numer rejestracyjny, numer VIN, rok produkcji, rodzaj spalanego paliwa. Pojedynczy samochód
			powiązany jest z użytkownikiem systemu, będącym tym samym jego właścicielem. Samochód jest wersjonowany, a atrybuty, których zmiana powoduje utworzenie nowej rewizji to: właściciel, numer rejestracyjny oraz rodzaj paliwa. 
			\hline
			SIssue 				&
		 	Nie 				&
		 	\textbf{SIssue} opisuje sytuację wyjątkową, związaną z użytkownikiem systemu. Pojedynczy obiekt tego typu opisany jest przez użytkownika systemu, który zgłosił problem oraz użytkownika będącego podmiotem
		 	problematycznej sytuacji. 
		 	\hline
			SPerson 			& 
			Tak 				& 
			Ten obiekt domenowy jest pojedynczą osobą w systemie. Obiekt tego typu nie jest tożsamy z użytkownikiem systemu. Atrybuty opisujące ten typ obiektu to: imię, nazwisko oraz dane kontaktowe. Zmiana któregokolwiek z tych	 atrybutów
			równoznaczna jest z utworzeniem nowej rewizji. 
			\hline
			SPersonContact 		&
			Nie 				&
			Obiekt tego typu odnosi się do danych kontaktowych, które związane są osobą. Taki obiekt opisany jest przez atrybut, który odnosi się do tego, czy jest to numer telefonu komórkowego lub email. Drugim atrybutem jest
			wartość danego kontaktu.
			\hline
			SReport 			&
			Nie					& 
			Obiekt będący częścią komponentu \textbf{RBuilder}. Dostarcza informacji o położeniu plików opisujących zarówno strukturę szablonu, jak i pomocnych w procesie generowania raportu.
			\hline
			SAuthority	 		&
			Nie					& 
			Obiekty tej klasy opisują uprawnienia jakie przypisane są pojedynczemu użytkowniki systemu. Dzięki nim możliwe jest kontrolowanie do jakich obszarów funkcjonalnych, pojedynczy użytkownik ma dostęp, a które
			rejony są dla niego niedostępne. 
			\hline
			SUser 				&
			Tak 				& 
			\textbf{SUser} jest faktycznym użytkownikiem aplikacji. Pojedynczy obiekt opisany jest przez: login, hasło, zestaw zmiennych logicznych pozwalających określić stan konta (konto nieaktywne, zablokowane lub nieważne). Rola użytkownika, to kim on jest w kontekście systemu, opisane jest przez zbiór uprawnień. Modyfikacja atrybutów odnoszących się do nazwy użytkownika lub hasła skutkuje utworzeniem nowej rewizji obiektu. 
			\hline
			SUserNotification 	& 
			Nie 				& 
			Powiadomienie jest wiadomością jaka jest wysyłana użytkowniki systemu. Składa się z treści wiadomości oraz daty jej wysłania. Dodatkowym atrybutem jest zmienna logiczna odnosząca się do faktu, czy adresat odczytał powiadomienie. 
		\end{longtable}
	\end{center}	
	
\subsection{Model - Warstwa repozytoriów}
	Repozytoria w aplikacji demonstracyjnej są niczym więcej jak jedynie zdefiniowanymi w odpowiedni sposób interfejsami. Odpowiedni sposób oznacza, że dla każdego z nich, pierwszym interfejsem w drzewie dziedziczenia jest \textbf{Repository}. Jest to kluczowe ponieważ w ten sposób szkielet aplikacji \textbf{Spring} rozpoznaje repozytoria podczas skanowania \textbf{classpath}. Programista jest zobowiązany jedynie, w przypadku chęci skorzystania, do utworzenia w swojej klasie pola typu tego interfejsu, a właściwa referencja do obiektu zostanie umieszczona poprzez \textbf{dependency injection}. Kod na listingu \ref{app:scarmaster_repo} pokazuję definicje repozytorium. 
	\begin{code}
		\inputminted[
			linenos=true,
			firstline=43,
			lastline=77,
			fontfamily=monospace,
			obeytabs=true,
			samepage=true,
			fontsize=\scriptsize
		]{java}{../SpringAtom_thesis/src/main/java/org/agatom/springatom/server/repository/repositories/car/SCarMasterRepository.java}
		% src file used
		\caption[\textbf{SCarMasterRepository} - interfejs repozytorium dla modelu \textbf{SCarMaster}]{
			\textbf{SCarMasterRepository} - interfejs repozytorium	 pokazujący użycie metod mapowanych
			na kwerendy oraz bardziej skomplikowanego zapytania z użyciem adnotacji \textbf{Query}, źródło: opracowanie własne
		}
		\label{app:scarmaster_repo}
	\end{code}
	 
	\paragraph{Rozszerzenie natywnej funkcjonalności repozytoriów} \hspace{0pt} \\
	Funkcjonalność repozytoriów dostarczona przez moduł \textbf{Spring Data JPA} (\ref{tech:spring_data_jpa}) została rozbudowana o kilka dodatkowych funkcji. Rozszerzenie dotyczyło operacji wykonywanych na obiektach wersjonowanych, a zdefiniowane zostało w klasie \textbf{SRepository}, widocznej na listingu \ref{app:srepository}. 	Dodatkowe funkcje pozwalają na wyszukiwanie obiektów domenowych według następujących kryteriów:
	\begin{itemize}
		\item konkretny klucz główny oraz konkretna rewizja (metoda \textbf{findInRevision(ID,N)}),
		\item konkretny klucz główny oraz konkretne rewizje (metoda \textbf{findInRevisions(ID,N...)}),
		\item konkretny klucz, gdzie modyfikacja nastąpiła w pewnym momencie czasu (metoda \textbf{findRevisions(ID,DateTime,TimeOperator)}),
		\item obliczenie ilości modyfikacji (metoda \textbf{countRevisions(ID)}
	\end{itemize}
	
	\begin{code}
		\inputminted[
			linenos=true,
			firstline=36,
			lastline=56,
			fontfamily=monospace,
			obeytabs=true,
			samepage=false,
			fontsize=\scriptsize
		]{java}{\SpringatomScrPath{/server/repository/SRepository.java}}
		% src file used
		\caption[\textbf{SRepository} - abstrakcyjne repozytorium wspierające dostęp do rewizji]{
			\textbf{SRepository} - abstrakcyjne repozytorium wspierające dostęp do rewizji, źródło: opracowanie własne
		}
		\label{app:srepository}
	\end{code}
	
	Uwzględnienie rozszerzenia nie byłoby możliwe bez modyfikacji procesu, podczas którego moduł \textbf{Spring Data JPA} tworzył obiekty implementujące funkcjonalność repozytoriów, zdefiniowaną poprzez interfejsy. Wprowadzony został dodatkowy element - fabryka, która nadpisywała oryginalny algorytm dostarczając implementacji zdefiniowanych w aplikacji praktycznej. Listing \ref{app:srepositories_factory_bean} pokazuje wycinek kodu klasy \textbf{SRepositoriesFactoryBean} realizujący tą czynność. 
	\begin{code}
		\inputminted[
			gobble=2,
			linenos=true,
			firstline=89,
			lastline=105,
			fontfamily=monospace,
			obeytabs=true,
			samepage=false,
			fontsize=\scriptsize
		]{java}{\SpringatomScrPath{/server/repository/factory/SRepositoriesFactoryBean.java}}
		\caption[\textbf{SRepositoriesFactoryBean} - fabryka repozytoriów dla implementacji własnej funkcjonalności]{
			\textbf{SRepositoriesFactoryBean} - fabryka repozytoriów dla implementacji własnej funkcjonalności, źródło: opracowanie własne
		}
		\label{app:srepositories_factory_bean}
	\end{code}
	
	\clearpage
	\paragraph{Lista repozytoriów} \hspace{0pt} \\
	Poniższa tabela zawiera listę repozytoriów zdefiniowanych w aplikacji demonstracyjnej, nazwę odpowiadającego jej obiektu domenowego oraz informację o tym, czy repozytorium wspiera dostęp do dziennika zmian tego obiektu.
	\begin{center}
		\begin{longtable}{| p{6cm} | p{5cm} | p{2cm} |}
			\caption[Lista repozytoriów danych]{Lista repozytoriów danych \label{mvc:repo_list}}\tabularnewline	
			
			% header
			\hline
				\multicolumn{1}{|c|}{\textbf{Repozytorium}} 			&
				\multicolumn{1}{|c|}{\textbf{Obiekt domenowy}} 		&
				\multicolumn{1}{|c|}{\textbf{Dziennik zmian}} 			\\
			\hline
			\endfirsthead
			
			\multicolumn{2}{c}
			{{\bfseries \tablename\ \thetable{} -- kontynuacja...}} 	\\
			\hline
				\multicolumn{1}{|c|}{\textbf{Repozytorium}} 			&
				\multicolumn{1}{|c|}{\textbf{Obiekt domenowy}} 		&
				\multicolumn{1}{|c|}{\textbf{Dziennik zmian}} 			\\
			\hline
			\endhead
				
			\hline
				\multicolumn{4}{|r|}{{Następna strona...}} \tabularnewline
			\hline
			\endfoot
	
			\hline
			\endlastfoot	
			% end of header
			
			% body starts her
			SAppointmentRepository & SAppointment & Nie \hline
			SAppointmentIssueRepository & SAppointmentIssue & Nie \hline
			SAppointmentTaskRepository & SAppointmentTask & Nie \hline
			SCarMasterRepository & SCarMaster & Nie \hline
			SCarRepository & SCar & Tak \hline
			SIssueRepository & SIssue & Nie \hline
			SPersonContactRepository & SPersonContacat & Nie \hline
			SPersonRepository & SPerson & Tak \hline
			SReportRepository & SReport & Tak \hline
			SUserAuthorityRepository & SUserAuthority & Nie \hline
			SUserNotificationRepository & SUserNotification & Nie \hline
			SUserRepository & SUser & Tak
		\end{longtable}
	\end{center}	
	
\subsection{Model - Warstwa serwisów}
	Serwisy	stanowią w aplikacji element realizujący logikę biznesową, jednak nie odnoszą się one jedynie do operacji
	na modelu danych. Również inne moduły aplikacji korzystają z własnych serwisów, jako miejsc gdzie funkcjonalność została
	zebrana i jest gotowa do użycia. 	
	\begin{code}
		\inputminted[
			linenos=true,
			firstline=33,
			lastline=45,
			fontfamily=monospace,
			obeytabs=true,
			samepage=false,
			fontsize=\scriptsize
		]{java}{\SpringatomScrPath{/server/service/domain/SPersonService.java}}
		% src file used
		\caption[\textbf{SPersonService} - interfejs serwisu dla modelu \textsc{SPerson}]{
			\textbf{SPersonService} - interfejs serwisu dla modelu \textsc{SPerson}, źródło: opracowanie własne
		}
		\label{app:sperson_service}
	\end{code}
	
	Serwisy zostały zaprojektowane aby korzystać z repozytoriów danych. Ma to swoje pozytywne skutki
	i nie jest wcale oznaką nadmiarowości kodu, czy też jego duplikowania. Z uwagi na fakt, ze serwisy
	odwołują się do danych poprzez interfejsy repozytoriów, podnosi to znacząco możliwości 
	późniejszych zmian w postaci silnika bazy danych. Dodatkową korzyścią jest zwiększenie możliwości testowania
	interesujących funkcji bez konieczności posiadania działającego połączenia z bazą danych.
	
\subsection{Widok - Warstwa widoku}
	Warstwa widoku została zaprojektowana z wykorzystaniem standardowej biblioteki tagów \textbf{JSTL}, plików \textbf{JSP} oraz biblioteki \textbf{Apache Tiles} \ref{tech:tiles}. Dzięki \textbf{Tiles} udało się zminimalizować zbędny kod w plikach \textbf{JSP} definiujący elementy, takie jak nagłówek strony, element \textit{<head>}. Do gotowego widoku można się było następnie odwoływać po unikatowej nazwie. 
	\begin{code}
		\inputminted[
			linenos=true,
			firstline=23,
			lastline=36,
			fontfamily=monospace,
			obeytabs=true,
			samepage=true,
			fontsize=\scriptsize
		]{xml}{../SpringAtom_thesis/src/main/webapp/ui/core/core-page-view.xml}
		% src file used
		\caption[Definicja \textit{tile} - podstawowego elementu widoku]{
			Definicja \textit{tile} - podstawowy element widoku w rozumieniu technologi Apache Tiles, źródło: opracowanie własne		
		}
		\label{app:apache_tiles_example}
	\end{code} 
	
	Listing \ref{app:apache_tiles_example} pokazuje deklarację abstrakcyjnej \textit{płytki}. Abstrakcyjność jest tutaj kwestią umowną ponieważ
	można by tę \textit{płytkę} zwrócić z dowolnego kontrolera i została by ona zrenderowana do poprawnego widoku HTML. Elementem, który jest w tej definicji
	zadeklarowany, lecz nie zdefiniowany, jest \textit{content} - faktyczna zawartość danej strony. Mimo to plik \textbf{JSP} odpowiadający tej płytce
	zawiera kod, który umieści ją w ostatecznej strukturze DOM. 
	
	\begin{code}
		\inputminted[
			linenos=true,
			firstline=51,
			lastline=53,
			fontfamily=monospace,
			obeytabs=true,
			samepage=true,
			fontsize=\scriptsize
		]{jsp}{../SpringAtom_thesis/src/main/webapp/ui/core/page.jsp}
		% src file used
		\label{app:apache_tiles_example_jsp}
		\caption[Fragment pliku \textbf{JSP} umieszczający w nim atrybut \textbf{content}]{
			Fragment pliku \textbf{JSP} umieszczający w nim atrybut \textbf{content}, źródło: opracowanie własne
		}
	\end{code}
	
	\paragraph{Generyczny kontroler zwracający widok} \hspace{0pt} \\
	Aby zminimalizować ilość kodu i pisania kontrolera posiadającego metody zwracające nazwy widoków dla każdego możliwego adresu, w aplikacji praktycznej istnieje tylko jeden kontroler wspierający to zadania. \textbf{SVTilesViewControler} działa w oparciu o plik, zawierający wpisy opisujące mapowania adresów na nazwy widoków. Adres jest tutaj kluczem, z którego kontroler buduje obiekt klasy \textbf{UriTemplate}. \textbf{UriTemplate} działa podobnie do wyrażenia regularnego i pozwala na stwierdzenie, czy adres pobrany z żądania, pasuje do szablonu. Listing \ref{app:svTilesViewController} pokazuje definicję metody realizującej mapowanie, natomiast listing \ref{app:mapping_props} to lista adresów wraz z odpowiadającymi im widokami.
	\begin{code}
		\inputminted[
			linenos=true,
			firstline=50,
			lastline=61,
			fontfamily=monospace,
			obeytabs=true,
			samepage=true,
			fontsize=\scriptsize
		]{java}{\SpringatomScrPath{/webmvc/controllers/SVTilesViewController.java}}
		% src file used
		\caption[Generyczny kontroler zwracający nazwy widoków]{
			Generyczny kontroler zwracający nazwy widoków zdefiniowanych w pliku \ref{app:mapping_props}, źródło: opracowanie własne
		}
		\label{app:svTilesViewController}
	}
	\end{code}
		\begin{code}
		\inputminted[
			linenos=true,
			firstline=18,
			lastline=34,
			fontfamily=monospace,
			obeytabs=true,
			samepage=true,
			fontsize=\scriptsize
		]{properties}{../SpringAtom_thesis/src/main/resources/org/agatom/springatom/webmvc/url-mapping.properties}
		% src file used
		\caption[Mapowania adresów na nazwy widoków]{
			Mapowania adresów na nazwy widoków, źródło: opracowanie własne
		}
		\label{app:mapping_props}
	\end{code}
		 
\subsection{Globalna - Warstwa konwersji}
	Warstwa konwersji typów jest modułem \textbf{Spring}, który wykorzystywany jest w momencie, kiedy z obiektu typu A programista chce uzyskać obiekt typu B. Dobrym przykładem jest najczęściej moment, w którym w kodzie JSP umieszczamy bezpośrednio nasz obiekt, korzystając z biblioteki tagów dostarczonej przez \textbf{Spring}. W tym momencie uruchamiany jest proces, mający na celu konwersję obiektu do reprezentatywnej postaci łańcucha znakowego, który można będzie wkomponować do drzewa DOM. Jest to rozwiązanie efektywne, niemniej posiadające jedno uchybienie. W przypadku, gdy programista chciałby uzyskać selektywny sposób, zależny od kontekstu, w jakim znajduje się obiekt lub też chęci uzyskania wartości jednego z atrybutów, nie jest on w stanie osiągnąć zamierzonego rezultatu, z uwagi na sposób, w jaki działa konwersja typów. Znalezienie pierwszego konwertera, który jest w stanie przeprowadzić żądaną transformację kończy proces wyszukiwania. Nie oznacza to wcale, że uzyskany wynik będzie zgodny z oczekiwanym. Z tego powodu aplikacja demonstracyjna rozszerza istniejącą funkcjonalność przez umożliwienie wybiórczego konwertowania między poszczególnymi typami. Na obecną chwilę zostało to zaimplementowane dla obiektów domenowych, a klasą kontrolującą selektywny proces jest \textbf{PersistableConverterPicker}.
	\begin{code}
		\inputminted[
			linenos=true,
			firstline=51,
			lastline=69,
			fontfamily=monospace,
			obeytabs=true, 
			samepage=true,
			fontsize=\scriptsize
		]{java}{\SpringatomScrPath{/server/model/conversion/picker/PersistableConverterPicker.java}}
		\caption[\textbf{PersistableConverterPicker} - koordynator selektywnej konwersji typów]{
			\textbf{PersistableConverterPicker} - koordynator selektywnej konwersji typów, źródło: opracowanie własne
		}
		\label{app:conversion_persistableToString}
	\end{code}
	\textbf{PersistableConverterPicker} posiada metody które pozwalają na wybór selektywnego konwertera do wykonania operacji
	transformacji. Zostało również zapewnione wsparcie dla istniejącej funkcjonalności, bez użycia selektora. Ta część realizowana jest w 
	metodzie \textbf{getDefaultConverter(...)}.
	\begin{code}
		\inputminted[
			linenos=true,
			firstline=72,
			lastline=99,
			fontfamily=monospace,
			obeytabs=true, 
			samepage=true,
			fontsize=\scriptsize
		]{java}{\SpringatomScrPath{/server/model/conversion/picker/PersistableConverterPicker.java}}
		\caption[\textbf{PersistableConverterPicker} - pobranie domyślnego konwertera dla typu]{
			\textbf{PersistableConverterPicker} - pobranie domyślnego konwertera dla typu, źródło: opracowanie własne
		}
		\label{app:conversion_defaultConverter}
	\end{code}
	
	\clearpage
	Selektywne konwertery różnią się od normalnych jedynie użyciem specjalnej adnotacji \textbf{PersistableConverterUtility} (listing \ref{app:conversion_PersistableConverterUtility}), która pozwala na ustalenie, że:
	\begin{itemize}
		\item konwerter jest domyślny, jeśli nie został zdefiniowany klucz,
		\item konwerter jest selektywny, jeśli istnieje zdefiniowany klucz.
	\end{itemize}
	
	\begin{code}
		\inputminted[
			linenos=true,
			firstline=34,
			lastline=52,
			fontfamily=monospace,
			obeytabs=true, 
			samepage=true,
			fontsize=\scriptsize
		]{java}{\SpringatomScrPath{/server/model/conversion/annotation/PersistableConverterUtility.java}}
		\caption[\textbf{PersistableConverterUtility} - adnotacja opisująca selektywny konwerter]{
			\textbf{PersistableConverterUtility} - adnotacja opisująca selektywny konwerter
		}
		\label{app:conversion_PersistableConverterUtility}
	\end{code}

\section{Przewodniki tworzenia nowych obiektów} 		\subsubsection{ReportBuilder - reporty biznesowe}
	\textbf{ReportBuilder} jest narzędziem będącym obecnie w fazie rozwoju, niemniej pozwalającym już teraz na tworzenie raportów biznesowych. Dedykowany dla użytkownika, daje mu możliwość wybrania zbioru interesujących go tabel, kolumn które zebrane razem stanowią logiczny zbiór używany w dalszej kolejności do konstrukcji zapytania do bazy danych i utworzenia gotowego raportu. Nie udało się znaleźć żadnego rozwiązania, które można by wykorzystać bezpośrednio w aplikacji WEB. Przewodnik składa się z 4 wymaganych kroków:
	\begin{enumerate}
		\item wybranie tabeli, dla której wygenerowany zostanie raport
		\item dostosowanie formatowania kolumn,
		\item podanie danych opisujących raport: tytuł, podtytuł, opis.
	\end{enumerate}		
	
	\begin{figure}[H]
		\centering
		\includegraphics[width=1.0\textwidth]{images/rbuilder_step1}
		\caption[Kreator nowego raportu - krok 1]{
			Kreator nowego raportu - krok 1
		}
		\label{app:wizard_newReport_step1}
	\end{figure}	
	\begin{figure}[H]
		\centering
		\includegraphics[width=1.0\textwidth]{images/rbuilder_step2}
		\caption[Kreator nowego raportu - krok 2]{
			Kreator nowego raportu - krok 2
		}
		\label{app:wizard_newReport_step2}
	\end{figure}		
	\begin{figure}[H]
		\centering
		\includegraphics[width=1.0\textwidth]{images/rbuilder_step3}
		\caption[Kreator nowego raportu - krok 3]{
			Kreator nowego raportu - krok 3
		}
		\label{app:wizard_newReport_step2}
	\end{figure}	
	
	\begin{wrapfigure}{r}{0.5\textwidth}
		\centering
		\includegraphics[width=1.0\textwidth]{images/rbuilder_generateReport}
		\caption[Generowanie nowego raportu - wybór docelowego formatu]{
			Generowanie nowego raportu - wybór docelowego formatu: \textbf{PDF}, \textbf{XLS}, \textbf{CSV}, \textbf{HTML}
		}
		\vspace{-10pt}
		\label{app:wizard_newReport_generateReport}
	\end{wrapfigure}
	
	Dzięki możliwości wykonywania akcji na poszczególnych wierszach tabeli, udało się przypisać dla każdego raportu link uruchamiający jego
	generowania oraz usunięcie. Usunięcie jest operacją trywialną z punktu widzenia użytkownika, ponieważ nie widzi on niczego poza 
	końcowym rezultatem. Z drugiej strony w momencie kliknięcia na przycisk \textbf{Generuj}, użytkownik ma możliwość
	wybrania końcowego formatu w jakim chciałby zobaczyć swoje dane. 
	Dalsza część funkcjonalności, czyli faktycznego zrenderowania gotowych danych została zaimplementowana z użyciem biblioteki wspierającej
	\textbf{JasperReports}, pochodzącą ze szkieletu aplikacji \textbf{Spring}. Przykładowy raport wygląda w następujący sposób:
	\begin{figure}[H]
		\centering
		\includegraphics[width=1.0\textwidth]{images/rbuilder_report}
		\caption[Gotowy raport utworzony przez komponent \textbf{RBuilder}]{
			Gotowy raport utworzony przez komponent \textbf{RBuilder}
		}
		\label{app:wizard_newReport_report}
	\end{figure}		

\pagebreak
\subsubsection{Kreator nowego użytkownika}
	Kreator nowego użytkownika pozwala na tworzenie nowy obiektów klasy \textbf{SUser}. Kwestia uprawnień jest tutaj szczególnie ważna z uwagi na to, że w przewodniku wybierany jest zestaw ról. Role, do których przypisany jest użytkownik, stanowią późniejszą bazę do weryfikacji dostępności funkcji systemu dla poszczególnych użytkowników. 
	\begin{figure}[H]
		\centering
		\includegraphics[width=1.0\textwidth]{images/newUser-basic}
		\caption[Kreator nowego użytkownika - dane podstawowe]{
			Kreator nowego użytkownika - dane podstawowe	
		}
		\label{app:newUser_basic}
	\end{figure}	
	\begin{figure}[H]
		\centering
		\includegraphics[width=1.0\textwidth]{images/newUser-roles}
		\caption[Kreator nowego użytkownika - uprawnienia użytkownika]{
			Kreator nowego użytkownika - uprawnienia użytkownika
		}
		\label{app:newUser_roles}
	\end{figure}	
	\begin{figure}[H]
		\centering
		\includegraphics[width=1.0\textwidth]{images/newUser-contacts}
		\caption[Kreator nowego użytkownika - dane kontaktowe]{
			Kreator nowego użytkownika - dane kontaktowe
		}
		\label{app:newUser_contacts}
	\end{figure}	

\subsubsection{Kreator nowego samochodu}
	Kreator nowego samochodu został zaprojektowany aby tworzyć nowe obiekty klasy \textbf{SCar}.	Pierwszym krokiem jest podanie \textbf{numeru VIN}, z którego system odczytuje wszelkie możliwe dane, które można odkodować korzystając z informacji dostępnych publicznie. Na obecną chwilę są to:
	\begin{itemize}
		\item rok produkcji, zwracany jako lista lat w których samochód mógł być wyprodukowany,
		\item kraj, w którym samochód został wyprodukowany.
	\end{itemize}
	
	W drugim kroku kreator nowego samochodu podaje takie informacje jak:
	\begin{itemize}
		\item markę oraz model,
		\item numer tablicy rejestracyjnej,
		\item rok produkcji,
		\item rodzaj paliwa,
		\item właściciela.
	\end{itemize}
	\begin{figure}[H]
		\centering
		\includegraphics[width=1.0\textwidth]{images/newCar-vin}
		\caption[Kreator nowego samochodu - numer VIN]{
			Kreator nowego samochodu - numer VIN
		}
		\label{app:newCar_vin}
	\end{figure}	
	\begin{figure}[H]
		\centering
		\includegraphics[width=1.0\textwidth]{images/newCar-data}
		\caption[Kreator nowego samochodu - pozostałe dane]{
			Kreator nowego samochodu - pozostałe dane
		}
		\label{app:newCar_data}
	\end{figure}

\subsubsection{Organizator spotkań}
	Organizator spotkań jest komponentem wspierającym zarządzania tym typem obiektów. 
	Pozwala na wgląd w listę wszystkich spotkań na 4 rożne sposoby:
	\begin{itemize}
		\item widok w kontekście wybranego dnia,
		\item widok w kontekście wybranego tygodnia,
		\item widok w kontekście wybranego miesiąca,
		\item widok tabelaryczny wszystkich spotkań.
	\end{itemize}
	
	\begin{figure}[H]
		\centering
		\includegraphics[width=1.0\textwidth]{images/calendarComponent-organizer}
		\caption[Komponent kalendarza wspierający organizację spotkań - organizer]{
			Komponent kalendarza wspierający organizację spotkań - organizer		
		}
		\label{app:component_calendar_organizer}
	\end{figure}	
	\begin{figure}[H]
		\centering
		\includegraphics[width=1.0\textwidth]{images/calendarComponent-table}
		\caption[Komponent kalendarza wspierający organizację spotkań - tabela]{
			Komponent kalendarza wspierający organizację spotkań - tabela	
		}
		\label{app:component_calendar_table}
	\end{figure}	
	
	Z poziomu organizera możliwe jest natomiast otwarcie strony domenowej dla wybranego spotkania, dostarczającej znacznie więcej informacji niż sam organizator oraz uruchomienie przewodnika utworzenia nowego spotkania. Ważną cechą tego przewodnika jest to, że zakres dat wybrany w organizatorze jest zachowany. Sam kreator został napisany w oparciu o technologię \textbf{Spring Web Flow}, dzięki czemu na poziomie jednego komponentu można wprowadzić dane takie jak: 
	\begin{itemize}
		\item mechanik, który będzie odpowiedzialny za wizytę,
		\item mechanik raportujący dane spotkanie, niemniej taką możliwość posiadają jedynie osoby uprawnione do tego,
		\item wybrany samochód,
		\item lista zadań do wykonania podczas wizyty,
		\item opcjonalny komentarz dla wizyty.
	\end{itemize}
	
	\begin{figure}[H]
		\centering
		\includegraphics[width=1.0\textwidth]{images/newAppointment_step1}
		\caption[Kreator nowego spotkania - krok 1]{
			Kreator nowego spotkania - krok 1
		}
		\label{app:wizard_newAppointment_step1}
	\end{figure}
	\begin{figure}[H]
		\centering
		\includegraphics[width=1.0\textwidth]{images/newAppointment_step2}
		\caption[Kreator nowego spotkania - krok 2]{
			Kreator nowego spotkania - krok 2
		}
		\label{app:wizard_newAppointment_step2}
	\end{figure}
	\begin{figure}[H]
		\centering
		\includegraphics[width=1.0\textwidth]{images/newAppointment_step3}
		\caption[Kreator nowego spotkania - krok 3]{
			Kreator nowego spotkania - krok 3
		}
		\label{app:wizard_newAppointment_step3}
	\end{figure}		
	
	Dane wejściowe są walidowane pod kątem logiki biznesowej:
	\begin{itemize}
		\item termin spotkania:
		\begin{itemize}
			\item podane godziny (oraz czas) muszą mieścić się w wybranym zakresie [7-21],
			\item koniec spotkania nie może nastąpić później niż jego początek,
			\item zbyt krótkie (30 minut) oraz zbyt długie (10 dni),
			\item pokrywa się z co najmniej jednym spotkaniem dla mechanika wybranego jako wykonawca
		\end{itemize} 
		\item wybrany samochód:
		\begin{itemize}
			\item jeśli z właścicielem samochodu powiązane są jakiekolwiek problematyczne informacje, wyświetlane jest ostrzeżenie
		\end{itemize}
	\end{itemize}
\section{Terminarz spotkań}								Organizator spotkań jest komponentem wspierającym zarządzanie tym typem obiektów. 
Pozwala na wgląd w listę wszystkich spotkań na 4 rożne sposoby:
\begin{itemize}
	\item widok w kontekście wybranego dnia,
	\item widok w kontekście wybranego tygodnia,
	\item widok w kontekście wybranego miesiąca,
	\item widok tabelaryczny wszystkich spotkań.
\end{itemize}

\begin{figure}[H]
	\centering
	\includegraphics[width=1.0\textwidth]{images/calendarComponent-organizer}
	\caption[Komponent kalendarza wspierający organizację spotkań - organizer]{
		Komponent kalendarza wspierający organizację spotkań - organizer, źródło: opracowanie własne			
	}
	\label{app:component_calendar_organizer}
\end{figure}
\begin{figure}[H]
	\centering
	\includegraphics[width=1.0\textwidth]{images/calendarComponent-table}
	\caption[Komponent kalendarza wspierający organizację spotkań - tabela]{
		Komponent kalendarza wspierający organizację spotkań - tabela, źródło: opracowanie własne	
	}
	\label{app:component_calendar_table}
\end{figure}	

Z poziomu organizera możliwe jest natomiast otwieranie strony domenowej dla wybranego spotkania, dostarczającej znacznie więcej informacji niż sam organizator oraz uruchomienie przewodnika utworzenia nowego spotkania. Ważną cechą tego przewodnika jest to, że zakres dat wybrany w organizatorze jest zachowany. Sam kreator został napisany w oparciu o technologię \textbf{Spring Web Flow}, dzięki czemu na poziomie jednego komponentu można wprowadzić dane takie jak: 
\begin{itemize}
	\item mechanik, który będzie odpowiedzialny za wizytę,
	\item mechanik raportujący dane spotkanie, niemniej taką możliwość posiadają jedynie osoby uprawnione do tego,
	\item wybrany samochód,
	\item lista zadań do wykonania podczas wizyty,
	\item opcjonalny komentarz dla wizyty.
\end{itemize}
\section{Generyczne moduły} 								Praca nad niewidoczną dla użytkownika końcowego częścią generycznych modułów zaowocowała opracowaniem 3 niezależnych modułów. To czym jest generyczny moduł, najłatwiej opisać na przykładzie jednej z metody programowania - programowania uogólnionego. Koncepcja ta oparta jest o założenie, że wiele fragmentów kodu jest niepotrzebnie powtarzanych tylko dlatego, że służą do przetwarzania różnych danych w identyczny sposób. Podejście generyczne pozwala na tworzenie ogólnych algorytmów, gdzie wymagania dotyczące danych, są nieokreślone lub zdefiniowane na poziomie abstrakcyjnych typów danych. 

\subsection{Strony obiektów domenowych oraz tabele}
	Podczas analizy wymagań warstwy biznesowej aplikacji demonstracyjnej stało się jasne, że podejście do reprezentacji danych w formie tabel, czy też wyświetlenia informacji o pojedynczym obiekcie domenowym, nie będzie operacją trywialną. Kluczowe było wybranie efektywnego sposobu reprezentacji obiektu lub obiektów domenowych w wybranym formacie. Z uwagi na przyjęte wymaganie funkcjonalne oraz w myśl zasady \textbf{DRY} (\ref{concept:dry}), tworzenie kolejnych klas z kodem odpowiedzialnym za powstawanie struktury tabeli lub strony internetowej było niedopuszczalne. Nie udało się również znaleźć żadnej biblioteki wspierającej taką funkcjonalność. Strony obiektów domenowych, zwane dalej \textbf{info page}, oraz tabele, to moduły generyczne ponieważ:
	\begin{itemize}
		\item algorytm pobierania danych jest niezależny od danych na poziomie typów danych, tj. możliwe było jego zaimplementowanie w abstrakcyjnej klasie generycznej, nie posiadającej na etapie kompilacji żadnej informacji o konkretnym typie obiektów, jakie klasa będzie przetwarzać,
		\item algorytm zdefiniowany jest raz,
		\item algorytm jest elastyczny,
		\item komponenty są w stanie samodzielnie skonwertować dane do typu prostego (liczba, wartość logiczna lub łańcuch znakowy), który można bezproblemowo zaprezentować po stronie klienta,
		\item finalne implementacje odpowiedzialne są jedynie za:
		\begin{itemize}
			\item dostarczenie informacji o budowie komponentu,
			\item dostarczenie wartości dla atrybutów dynamicznych, nie związanych z typem przetwarzanych danych
		\end{itemize}		 
	\end{itemize}
	
	Artefaktem definiującym powyższe założenie, będącym korzeniem całej hierarchii klas tych komponentów, jest interfejs \textbf{ComponentBuilder} (listening \ref{app:component_builder}). To właśnie ten interfejs jest następnie używany w kontrolerach odpowiedzialnych za mapowanie żądań pobrania definicji lub danych z poszczególnych komponentów. Same kontrolery nie zależą bezpośrednio od konkretnych implementacji, ale opierają się na \textbf{API} wspólnym dla wszystkich obiektów typu \textbf{ComponentBuilder}. Szczególnie ważne są tutaj następujące metody:
	\begin{itemize}
		\item \textbf{getDefinition()} - zadaniem tej metody jest zwrócenie struktury, definicji danego komponentu,
		\item \textbf{getData()} - zadaniem tej metody jest zwrócenie danych danego komponentu.
	\end{itemize}
	
	\begin{code}
		\inputminted[
			linenos=true,
			firstline=32,
			lastline=46,
			fontfamily=monospace,
			obeytabs=true,
			samepage=true,
			fontsize=\scriptsize
		]{java}{\SpringatomScrPath{/web/component/builders/ComponentBuilder.java}}
		\caption[\textbf{ComponentBuilder - korzeń hierarchii modułu komponentów}]{
			\textbf{ComponentBuilder} - korzeń hierarchii modułu komponentów, który opisuje rolę tego rodzaju obiektów w systemie, źródło: opracowanie własne				
		}
		\label{app:component_builder}
	\end{code}
	
	\paragraph{Problem wielu zapytań AJAX} \hspace{0pt} \\
	Aplikacja internetowa korzysta z zapytań do serwera aby pobierać treść, dane oraz zasoby, które mogą zostać zaprezentowane użytkownikowi. Z każdym zapytaniem wiążą się jednak koszty, których nie sposób jest uniknąć. Narzut związany z inicjacją zapytania, odpowiedzią serwera, jej wielkością oraz samym czasem odpowiedzi jest kwestią, którą należy uwzględnić. Problemem, który pojawił się podczas projektowania algorytmu renderowania \textbf{info page} oraz tabeli wynikał z uogólnionej koncepcji według, której miały one działać. Niemniej, bez wiedzy o typie obiektu domenowego, którego atrybuty miały być przedstawione na stronie lub którego obiekty miały być wyświetlone w tabeli, nie było możliwości wstępnego przygotowania struktury takiego komponentu. Jedną z możliwości rozwiązania tego problemu było wykonanie dwóch oddzielnych zapytań do serwera, najpierw po definicję, a następnie po dane. Inną ewentualnością było wystosowanie pojedynczego zapytania, gdzie odpowiedź miałaby zawierać informacja o charakterystyce i danych. W drugim przypadku kwestią dyskusyjną stawał się rozmiar odpowiedzi, a także format w jakim przedstawiona musiałaby być definicja, aby możliwe było jej efektywne przetworzenie na odpowiednią strukturę DOM.  
	
	Wyjściem z sytuacji było skorzystanie z możliwości dynamicznego definiowania zawartości stron HTML, budowanych poprzez pliki JSP. Idea zakładała wykorzystanie mechanizmów dostępnych tylko i wyłącznie po stronie serwera, takich jak \textbf{tagi JSTL} i serwisy aplikacji, do utworzenia struktury DOM i odesłania gotowego dokumentu w odpowiedzi na żądania. Alternatywą byłoby budowanie kolejnych elementów strony przez \textbf{JavaScript}, co, nawet przy wykorzystaniu bibliotek takich jak \textbf{jQuery}, byłoby operacją mozolną i podatną na błędy. W kontekście omawianych komponentów, wybrane podejście zostało zastosowane zarówno dla stron obiektów domenowych, jak i tabel. Jedynie w przypadku \textbf{info page}, również dane, umieszczane są w strukturze dokumentu po stronie serwera. 
	
	\paragraph{Problem źródła danych} \hspace{0pt} \\
	Główna funkcjonalność tej części aplikacji demonstracyjnej sprowadza się do wydajnej i rozszerzalnej zmiany reprezentacji pewnej informacji na inną. Nie byłoby to możliwe bez źródła danych. Problemem z repozytorium, które byłoby w stanie sprostać zadaniu pobrania obiektów konkretnego typu, sprowadzał się właśnie do typu. W myśl idei jaka przyświeca programowaniu uogólnionemu, to nie dane są ważne, ale algorytm ich przetwarzania. Niemniej, dzięki wsparciu biblioteki \textbf{Spring Data JPA} (\ref{tech:spring_data_jpa}), kwestia ta została łatwo rozwiązania. Wspomniane repozytoria, w kontekście \textbf{Spring Data JPA}, są niczym innym jak generycznymi interfejsami. Wykorzystując ten fakt oraz własność typów generycznych języka Java, gdzie typ uogólniony, znajdujący się w deklaracji klasy, może zostać użyty w polach oraz metodach tej klasy (listining \ref{app:generic_types_matching}), repozytoria zdefiniowane zostały jako pola poszczególnych komponentów. Ostatecznie, korzystając z funkcjonalność szkieletu aplikacji \textbf{Spring}, pozwalając na wstrzykiwania zależności generycznych klas. Zachowane zostało bezpieczeństwo typów, a generyczne algorytmy uzyskały możliwość korzystania ze źródła danych.
	\begin{code}
		\inputminted[
			linenos=true,
			firstline=69,
			lastline=76,
			fontfamily=monospace,
			obeytabs=true,
			samepage=false,
			fontsize=\scriptsize
		]{java}{\SpringatomScrPath{/web/component/builders/table/TableComponentBuilder.java}}
		\caption[Typy generyczne w deklaracji oraz polach klasy w języku Java]{
			Typy generyczne w deklaracji oraz polach klasy w języku Java, źródło: opracowanie własne	
		}
		\label{app:generic_types_matching}
	\end{code}
	\paragraph{Problem identyfikacji \textbf{ComponentBuilder}} \hspace{0pt} \\
	Komponenty zostały tak zaprojektowane, aby zminimalizować poziom zależności między konkretnymi instancjami \textbf{ComponentBuilder} oraz odpowiadającym im elementów interfejsu użytkownika. Problematyczne okazało się zdefiniowanie powiązania pozwalającego nawigować między oboma artefaktami. Rozwiązanie oparte zostało na adnotacjach języka Java oraz zdolności szkieletu aplikacji \textbf{Spring} do wybierania obiektów opatrzonych konkretnymi adnotacjami spośród wszystkich zdefiniowanych.  
	\begin{code}
		\inputminted[
			linenos=true,
			firstline=40,
			lastline=53,
			fontfamily=monospace,
			obeytabs=true,
			samepage=false,
			fontsize=\scriptsize
		]{java}{\SpringatomScrPath{/web/component/builders/annotation/ComponentBuilds.java}}
		\caption[\textbf{ComponentBuilds} - adnotacja opisująca \textbf{omponentBuilder}]{
			\textbf{ComponentBuilds} - adnotacja opisująca \textbf{ComponentBuilder}, źródło: opracowanie własne
		}
		\label{app:componentBuilds_annotation}
	\end{code}	
	
	\textbf{ComponentBuilds} pozwala na zdefiniowanie następujących informacji:
	\begin{itemize}
		\item unikatowy klucz pod którym obiekt istnieje w aplikacji,
		\item typ budowanego komponentu: strona lub tabela
	\end{itemize}
	Dzięki \textbf{EntityBased} dany komponent zostaje ściśle powiązany z pewną klasą biznesowego modelu danych, jako artefakt zdolny utworzyć reprezentację obiektu lub obiektów tej klasy. Wszystkie te informacje pozwalają na pobranie obiektu \textbf{ComponentBuilder} skonkretyzowanego na przygotowanie tabeli, której kolejne wiersze odpowiadają konkretnym obiektom domenowym lub gotowego na utworzenie definicji strony domenowej.  
	\begin{code}
		\inputminted[
			linenos=true,
			firstline=27,
			lastline=33,
			fontfamily=monospace,
			obeytabs=true,
			samepage=false,
			fontsize=\scriptsize
		]{java}{\SpringatomScrPath{/web/component/builders/annotation/EntityBased.java}}
		\caption[\textbf{EntityBased} - adnotacja opisująca klasę obiektu domenowego, z którą związana jest konkretny \textbf{ComponentBuilder}]{
			\textbf{EntityBased} - adnotacja opisująca klasę obiektu domenowego, z którą związana jest konkretny \textbf{ComponentBuilder}, źródło: opracowanie własne
		}
		\label{app:entityBased_annotation}
	\end{code}	

\subsection{InfoPage - strony obiektów modelu danych}
	\textbf{InfoPage} jest komponentem budującym reprezentację obiektu domenowego jako strony internetowej. Składają się na niego artefakty opisujące strukturę strony, skonkretyzowana, pod kątem przetwarzania danych, implementacja interfejsu \textbf{ComponentBuilder} (\ref{app:component_builder}), klasy pomocnicze, usprawniające proces budowania struktury i pozyskiwania definicji komponentów na podstawie unikatowego klucza, klasy obiektu domenowego związanej z daną stroną lub adresu wpisanego w oknie przeglądarki. 
	
	\paragraph{Definicja komponentu InfoPage} \hspace{0pt} \\
	Pełna definicja \textbf{InfoPage} składa się z dwóch elementów: rozszerzenia klasy \textbf{EntityInfoPageComponentBuilder} i \textbf{SEntityInfoPage}. \textbf{SEntityInfoPage} (listining \ref{app:sEntityInfoPage}) dostarcza metadanych o:
	\begin{itemize}
		\item klasie obiektu domenowego,
		\item unikatowym identyfikatorze strony,
		\item unikatowym kluczu będącym częścią adresu danej strony
	\end{itemize}
	Podczas startu, aplikacja pobiera konkretne implementacje \textbf{SEntityInfoPage}, które istnieją w programie jako obiekty, dzięki wsparciu szkieletu aplikacji \textbf{Spring}, który automatycznie tworzy instancje opatrzonych odpowiednią adnotacją (\textbf{Component}, \textbf{Service} lub \textbf{Repository}). Dzięki temu procesowi możliwe jest odwoływanie się do meta obiektu \textbf{InfoPage} poprzez dowolny atrybut opisujący go. 
	\clearpage
	\begin{code}
			\inputminted[
				linenos=true,
				firstline=25,
				lastline=32,
				fontfamily=monospace,
				obeytabs=true,
				samepage=false,
				fontsize=\scriptsize
			]{java}{\SpringatomScrPath{/web/infopages/SEntityInfoPage.java}}
			\caption[\textbf{SEntityInfoPage} - interfejs opisujący metadane związane ze stroną domenową]{
				\textbf{SEntityInfoPage} - interfejs opisujący metadane związane ze stroną domenową, źródło: opracowanie własne
			}
			\label{app:sEntityInfoPage}
	\end{code}	
	
	\paragraph{Algorytm pobrania danych dla strony domenowej} \hspace{0pt} \\
	\textbf{EntityInfoPageComponentBuilder} jest implementacją \textbf{ComponentBuilder} (\ref{app:component_builder}) zawierającą skonkretyzowany algorytm pobierania wartości atrybutów, będących częścią danej strony. Jego zadaniem jest również przetworzenie danych na reprezentacją, którą będzie można wyświetlić po stronie klienta, jako prosty łańcuch znakowy lub liczbę. Widoczna w interfejsie \textbf{ComponentBuilder} metoda \textbf{getDefinition} pozostaje na tym poziomie nie zaimplementowana. Przyczyna tego leży w odmienności poszczególnych stron obiektów domenowych, sprowadzającej się do różnej ich struktury, różnych atrybutów, które będą widoczne i ostatecznie do innej klasy biznesowego modelu danych, skojarzonej z komponentem \textbf{InfoPage}.
	
	\subparagraph{Rozwiązanie problemu reprezentacji danych} \hspace{0pt} \\
	W przypadku strony domenowej atrybutami, które wymagały rozwiązania niespójności typów danych, były relacje klucz obcy - klucz główny. Sytuacja kiedy obiekt, będący podmiotem danej strony, zależy od innego obiektu, została rozwiązania jako przedstawienie takiej zależności w postaci hiperłącza do strony domenowej, jeśli takowa została zdefiniowana (innymi słowy, jeśli w systemie można było znaleźć definicję klasy \textbf{SEntityInfoPage} oraz odpowiadającej jej implementacji \textbf{EntityInfoPageComponentBuilder}. Odwrotna sytuacja miała miejsce, kiedy podmiot strony był zależnością dla innych obiektów domenowych. Rozwiązana została ona poprzez przekazanie, jako wartości dla atrybutu, informacji niezbędnych do załadowania tabeli. Analogicznie do pierwszego przypadku, również w tym momencie, koniecznie było najpierw ustalenie, czy w aplikacji istnieje odpowiednia implementacja klasy \textbf{TableComponentBuilder}.
	Problem nieczytelnej wartości typów wyliczeniowych został rozwiązany z wykorzystaniem specjalnego pliku, zawierającego listę wszystkich wartości wszystkich typów wyliczeniowych wraz z ich czytelną wartością. 
	 
	\begin{code}
			\inputminted[
				linenos=true,
				firstline=98,
				lastline=149,
				fontfamily=monospace,
				obeytabs=true,
				samepage=false,
				fontsize=\scriptsize
			]{java}{\SpringatomScrPath{/web/infopages/component/builder/EntityInfoPageComponentBuilder.java}}
			\caption[\textbf{EntityInfoPageComponentBuilder} - implementacja pozyskania danych dla strony obiekty domenowego]{
				\textbf{EntityInfoPageComponentBuilder} - implementacja pozyskania danych dla strony obiekty domenowego, źródło: opracowanie własne
			}
			\label{app:entityInfoPageComponentBuilder}
	\end{code}	
	
	Utworzenie nowej strony sprowadza się sprowadza się do zaimplementowania \textbf{EntityInfoPageComponentBuilder}, oznaczenia nowej klasy adnotacjami: \textbf{ComponentBuilds} (listing \ref{app:componentBuilds_annotation}) i \textbf{EntityBased} (listing \ref{app:entityBased_annotation}) oraz utworzenia nowej implementacji interfejsu \textbf{SEntityInfoPage}. 
	\clearpage
	\paragraph{Proces renderowania strony domenowej} \hspace{0pt} \\
	\begin{figure}[H]
		\centering
		\includegraphics[width=1.0\textwidth]{images/infoPage}
		\caption[Strona domenowa dla spotkania]{
			Strona domenowa dla spotkania, źródło: opracowanie własne	
		}
		\label{app:infoPage}
	\end{figure}
	
	Na rysunku \ref{app:infoPage} pokazana została strona domenowa obiektu opisującego pojedyncza wizytę w warsztacie samochodowym. Zdefiniowane zostały 3 panele z atrybutami. Panel podstawowy oraz dwa panele opisujące relacje zachodzące między spotkaniem a innymi obiektami. Spotkanie posiada więc listę zadań oraz zostało przypisane do pewnego samochodu, zgłoszone i wykonane przez konkretnych mechaników. Listing kodu \ref{app:infoPageSrcCode} przedstawia fragmentu kodu Java odpowiedzialny za zbudowania struktury strony.
	\begin{code}
		\inputminted[
			linenos=true,
			firstline=50,
			lastline=75,
			fontfamily=monospace,
			obeytabs=true,
			samepage=false,
			fontsize=\scriptsize
		]{java}{\SpringatomScrPath{/webmvc/pages/builders/AppointmentInfoPageComponentBuilder.java}}
		\caption[Strona domenowa dla spotkania - kod źródłowy]{Strona domenowa dla spotkania - kod źródłowy, źródło: opracowanie własne}
		\label{app:infoPageSrcCode}
	\end{code}
	
	Proces, w wyniku którego wywołana została metoda z listingu \ref{app:infoPageSrcCode}, tworząca strukturę strony, oraz wywołujący metodą pokazaną na listingu \ref{app:entityInfoPageComponentBuilder}}, która zwraca dane dla konkretnych atrybutów, ilustruje poniższa tabela:
	\begin{center}
		\begin{longtable}{| p{2.5cm} | p{13cm} |}
			\caption[Proces renderowania strony domenowej]{Proces renderowania strony domenowej}\tabularnewline	
			
			% header
			\hline
				\multicolumn{1}{|c|}{\textbf{Krok}} 		&
				\multicolumn{1}{|c|}{\textbf{Opis}} 		\tabularnewline
			\hline
			\endfirsthead
			
			\multicolumn{2}{c}
			{{\bfseries \tablename\ \thetable{} -- kontynuacja...}} \tabularnewline
			\hline
				\multicolumn{1}{|c|}{\textbf{Krok}} 		&
				\multicolumn{1}{|c|}{\textbf{Opis}} 		\tabularnewline
			\hline
			\endhead
				
			\hline
				\multicolumn{2}{|r|}{{Następna strona...}} \tabularnewline
			\hline
			\endfoot
	
			\hline
			\endlastfoot	
			% end of header
			
			% body starts her
			\textbf{1}							&
			Zdarzeniem inicjującym proces renderowania strony domenowej jest wpisanie w przeglądarce adresu, 
			który zostaje rozpoznany przez aplikację jako wskazujący na komponent \textbf{InfoPage}. Jest to możliwe ponieważ 
			adres komponentu posiadana określony format: \textbf{/ip/\{klucz strony\}/\{klucz główny\}/\{wersja\}}. Najważniejszym elementem
			adresu jest \textbf{klucz strony} (listining \ref{app:sEntityInfoPage}). 
			\tabularnewline
			\hline
			\textbf{2}							&
			Żądanie zostaje zmapowane do metody \textbf{getInfoPageView} (listining \ref{app:svInfoPageController}). Jej zadaniem jest
			pobranie meta danych strony domenowej na podstawie unikatowego klucza strony, wspomnianego w punkcie 1. Obiekt klasy \textbf{SInfoPage}
			zostaje umieszczony w odpowiedzi razem z hiperłączem, pod którym dostępny będzie dokument HTML zawierający gotową stronę \textbf{InfoPage}.
			\tabularnewline
			\hline
			\textbf{3}							&
			Po stronie klienta, kod JavaScript zajmuje się przygotowaniem asynchronicznego żądania po gotowy widok strony, dostępny pod adresem
			z punktu 2.
			\tabularnewline
			\hline
			\textbf{4}							&
			Żądanie, wysłane z punkcie 3, zostaje zmapowane do metody \textbf{getInfoPageViewData} (listining \ref{app:svInfoPageController}). Metoda pobiera instancję
			klasy \textbf{ComponentBuilder}, która spełnia następujące wymagania:
			\begin{itemize}
				\item buduje strukturę strony domenowej,
				\item związana jest z typem obiektu domenowego
			\end{itemize}
			Znaleziony \textbf{ComponentBuilder} jest następnie umieszczany w odpowiedzi razem z widokiem, będącym szablonem (listing \ref{app:domain_render_jsp}) przyszłej strony.
			Szablon zaprojektowany został dla jednoczesnego tworzenia dokumentu HTML, zawierającego atrybuty oraz odpowiadające im wartości. 
			To właśnie z poziomu pliku JSP, w przypadku stron obiektów domenowych, wywoływane są metody 
			interfejsu \textbf{ComponentBuilder} pozwalające na pobranie jej struktury oraz danych. 
		\end{longtable}
	\end{center}
	
	\begin{code}
		\inputminted[
			linenos=true,
			firstnumbe=59,
			firstline=59,
			lastline=95,
			fontfamily=monospace,
			obeytabs=false,
			samepage=false,
			fontsize=\scriptsize
		]{java}{\SpringatomScrPath{/webmvc/controllers/SVInfoPageController.java}}
		\caption[SVInfoPageController - kontroler obsługujący żądania komponentu \textbf{InfoPage}]{SVInfoPageController - kontroler obsługujący żądania komponentu \textbf{InfoPage}}
		\label{app:svInfoPageController}
	\end{code}

	\begin{code}
		\inputminted[
			gobble=5,
			linenos=true,
			firstnumbe=64,
			firstline=64,
			lastline=91,
			fontfamily=monospace,
			obeytabs=false,
			samepage=false,
			fontsize=\scriptsize
		]{jsp}{../SpringAtom_thesis/src/main/webapp/ui/ip/_render.jsp}
		\caption[Szablon strony obiektu domenowego]{Fragment szablonu strony obiektu domenowego, tworzący ciało tabeli}
		\label{app:domain_render_jsp}
	\end{code}
	
\subsection{TableBuilder}
	Rzadko zdarza się żeby aplikacja nie wymagała korzystania z tabel do prezentowania danych. Są one szczególnie użyteczne zwłaszcza w momencie, kiedy ilość możliwych obiektów do jednorazowego wyświetlenia sięga co najmniej kilkudziesięciu elementów. Każda z nich opisana jest zarówno przez zbiór danych, jak i przez zbiór kolumn. Aby poprawnie zdefiniować tabelę wymagane jest podanie obu zbiorów. Istnieją dwie całkowicie odmienne koncepcje tego zagadnienia: statyczna oraz dynamiczna. W przypadku podejścia statycznego, struktura tabeli jest ręcznie umieszczana w strukturze DOM dokumentu HTML. Takie podejście sprawia jednak, że tabela jest stała, zarówno pod kątem tego co wyświetla, jak również w jaki sposób to robi, każda modyfikacja zbioru danych, wymaga więc ręcznej aktualizacji. Dynamiczną realizację tej kwestii można podzielić na przypadki, gdzie:
	\begin{enumerate}
		\item struktura jest statyczna, źródło danych jest dynamiczne,
		\item struktura oraz dane są dynamiczne
	\end{enumerate}
	
	\paragraph{Problem reprezentacji danych} \hspace{0pt} \\
	Problematyczne w obu podejściach nie okazuje się jednak ani wprowadzenie schematu mapującego poszczególne kolumny na odpowiadające im pola danych, ani też automatyzacja procesu wpisującego poprawną, w rozumieniu języka HTML, definicję tabeli do struktury DOM dokumentu. W przypadku pliku JSP rozwiązanie wymagałoby by ustalenia zbioru kolumn, pobrania danych z repozytorium, ręcznego wpisania statycznych elementów tabeli, takich jak na przykład nagłówek i utworzenia ciała tabeli (kolejnych wierszy) na podstawie zbioru danych. Inne podejście zakładałoby wykorzystanie biblioteki JavaScript dostarczającej dodatkowej warstwy abstrakcji nad procesem definiowania tabeli oraz zdolnej do wygenerowania zapytania Ajax pod wskazany adres, pod którym dostępny byłby zbiór danych. Problemem jest reprezentacja danych. Poniższe zestawienie opisuje wybrane przypadki, gdzie pokazane zostały potencjalne rozwiązania, a rzeczywista i oczekiwana reprezentacja danych jest od siebie różna:
	
	\begin{center}
		\begin{longtable}{| p{4cm} | p{12cm} |}
			\caption[Zestawienie problemów reprezentacji danych dla tabel]{
				Zestawienie problemów reprezentacji danych dla tabel	
				\label{app:tablesProblems}
			}\tabularnewline	
			
			% header
			\hline
				\multicolumn{1}{|c|}{\textbf{Problem}} 						&
				\multicolumn{1}{|c|}{\textbf{Potencjalne rozwiązanie}} 		\tabularnewline
			\hline
			\endfirsthead
			
			\multicolumn{2}{c}
			{{\bfseries \tablename\ \thetable{} -- kontynuacja...}} \tabularnewline
			\hline
				\multicolumn{1}{|c|}{\textbf{Problem}} 						&
				\multicolumn{1}{|c|}{\textbf{Potencjalne rozwiązanie}} 		\tabularnewline
			\hline
			\endhead
				
			\hline
				\multicolumn{2}{|r|}{{Następna strona...}} \tabularnewline
			\hline
			\endfoot
	
			\hline
			\endlastfoot	
			% end of header
			
			% body starts her
			\textbf{Relacja klucz główny - klucz obcy}							&
			Wykonanie następnego zapytania do repozytorium danych, pobranie powiązanego obiektu. Kolejną
			niewiadomą w tym przypadku jest niestety to, jaki atrybut powiązanego obiektu wybrać, który
			jednoznacznie by go opisywał, a w pierwszej kolejności wiedza, które z atrybutów należy
			interpretować jako relacje. 
			\tabularnewline
			\hline
			\textbf{Data jako znacznik czasowy}								&
			Przechowanie daty jako znaczników czasowych jest zalecane, ponieważ jest to rozwiązanie przenośne,
			pozwalające na uniknięcie problemu operowania na datach jako łańcuchach znakowych, w których format
			zapisu jest kwestią zależną od programisty i podatną na błędy. Utworzenie zrozumiałej reprezentacji
			wymagałoby by wiedzy o tym, które kolumny krotki bazy danych, to tak naprawdę daty.
			\tabularnewline
			\hline
			\textbf{Typy wyliczeniowe}											&
			\textbf{Typ wyliczeniowy}, w kontekście języka Java, to specjalny rodzaj klasy, pozwalający na tworzenie
			stałych w trakcie kompilacji programu, które, w sposób niezmienny, opisują pewną właściwość obiektu, z 
			którą są związane. Na poziomie bazy danych możliwe jest przechowanie wartości typu wyliczeniowego jako
			łańcucha znakowego. Problemem jest tutaj format w jakim kolejne pozycje typu wyliczeniowego są definiowane.
			Najczęściej pisane dużymi literami, nie posiadając spacji nie są dobrym kandydatem do bezpośredniego
			zaprezentowania w tabeli. Ponownie, rozwiązanie tego problemu, wymagałoby wiedzy nie tyle o tym, które
			z atrybutów należy interpretować jako pozycje typu wyliczeniowego, ale także tego, o który typ
			wyliczeniowy chodzi oraz danych, opisujących jego wartość w sposób zrozumiały dla użytkownika. 
		\end{longtable}
	\end{center}
	
	\paragraph{TableComponentBuilder} \hspace{0pt} \\
	\textbf{TableComponentBuilder} jest implementacją interfejsu \textbf{ComponentBuilder} (listing \ref{app:componentBuilds_annotation}), który został zaprojektowany do budowany definicji oraz zbioru danych dla tabel. Rozwiązuje on problemy zdefiniowane w tabeli \ref{app:tablesProblems} oraz dostarcza sposobu wsparcia następujących aspektów pracy z tabelami:
	\begin{itemize}
		\item wsparcie dla sortowania po stronie serwera,
		\item wsparcie dla filtrowania po stronie serwera,
		\item wsparcie dla dynamicznych kolumn, nie związanych bezpośrednio z typem obiektów, które tabela wyświetla
	\end{itemize}
	Działa on w oparciu o bibliotekę \textbf{Dandelion Datatables} (\ref{tech:dandelion}). Wykorzystanie biblioteki \textbf{Dandelion}, pozwoliło na ograniczenie ilości kodu niezbędnego do poprawnego wpisania definicji tabeli do struktury dokumentu HTML. Odpowiedzialność za ten proces została oddelegowana do wykorzystanej biblioteki, podobnie jak wykonanie asynchronicznego zapytania pod adres, pod którym znajduje się zbiór danych dla tabeli.
	
	\paragraph{Algorytm budowania zbioru danych} \hspace{0pt} \\
	Algorytm odpowiedzialny za utworzeniu zbioru danych jest podobny do tego, który zdefiniowano dla \textbf{InfoPage} (listing \ref{app:entityInfoPageComponentBuilder}). Różnice polegają na formacie w jakim dane są zwracane oraz w sposobie generowania zapytania do repozytorium danych. Podczas gdy strona domenowa była związana z pojedynczym obiektem, tabela związana jest z pewną ich liczbą. Dlatego też algorytm został skonkretyzowany na zwrócenia danych jako kolekcji, gdzie kolejne jej elementy odpowiadają kolejnym wierszom tabeli. Uwzględnia on także stronnicowanie, które polega na zwróceniu jedynie pewnego wycinka danych, zamiast wszystkich możliwych rekordów z bazy danych.
	\clearpage
	\begin{code}
		\inputminted[
			gobble=4,
			linenos=true,
			firstline=117,
			lastline=151,
			fontfamily=monospace,
			obeytabs=false,
			samepage=false,
			fontsize=\scriptsize
		]{java}{\SpringatomScrPath{/web/component/builders/table/TableComponentBuilder.java}}
		\caption[\textbf{TableComponentBuilder} - fragment implementacji pozyskania zbioru danych dla tabel]{\textbf{TableComponentBuilder} - fragment implementacji algorytmu pozyskania zbioru danych dla tabel}
		\label{app:tableComponentBuilder}
	\end{code}
	
	Warto w tym miejscu wspomnieć o podwójnym trybie działania algorytmu. W momencie, gdy tabela działa w kontekście komponentu \textbf{InfoPage}, nadrzędnym obiektem, zwanym dalej kontekstem, jest obiekt z którym skojarzona jest strona obiektu domenowego. Dla tego przypadku, tabela zwraca zbiór danych, który dla wspomnianego kontekstu jest wynikiem transformacji zależności klucz główny - klucz obcy. Z drugiej strony, komponent tabeli może zostać umieszczony na podstronie, gdzie kontekst nie jest zdefiniowany. W takim wypadku, budowany zbiór danych nie jest zawężany do żadnej relacji między poszczególnymi obiektami. Widoczne na listingu \ref{app:tableComponentBuilder} wywołanie metody \textbf{getPridacate} wykorzystuje funkcję \textbf{isInContext} (listing \ref{app:tcd_isInContext}), która rozpoznaje czy kontekst jest obecny i tym samym decyduje o zachowaniu tabeli. Jeśli kontekst istnieje tworzony jest predykat. Zadanie to realizowane jest na poziomie finalnych implementacji klas.
		
	\begin{code}
		\inputminted[
			gobble=1,
			linenos=true,
			firstline=290,
			lastline=293,
			fontfamily=monospace,
			obeytabs=false,
			samepage=false,
			fontsize=\scriptsize
		]{java}{\SpringatomScrPath{/web/component/builders/table/TableComponentBuilder.java}}
		\caption[\textbf{isInContext} - metoda \textbf{TableComponentBuilder} określająca tryb pracy]{
			\textbf{isInContext} - metoda \textbf{TableComponentBuilder} określająca tryb pracy komponentu dla algorytmu budowania zbioru danych, źródło: opracowanie własne
		}
		\label{app:tcd_isInContext}
	\end{code}
	
	\paragraph{Proces renderowania tabeli} \hspace{0pt} \\
	\begin{figure}[hb]
		\centering
		\includegraphics[width=1.0\textwidth]{images/table_inContext}
		\caption[Tabela wyświetlająca wszystkie pojazdy danej marki i modelu]{
			Tabela wyświetlająca wszystkie pojazdy danej samochodu \textbf{Fiat Panda}, źródło: opracowanie własne
		}
		\label{app:tableInContext}
	\end{figure}	
	
	Tabela na rysunku \ref{app:tableInContext} odpowiada implementacja klasy \textbf{TableComponentBuilder} widocznej na listingu \ref{app:cars_table_src_code}. Posiada ona 5 kolumn, z których 4 utworzone zostały w metodzie \textbf{buildDefinition} klasy \textbf{CarsTableBuilder}. Ostatnia kolumna dodawana jest domyślnie do każdej tabeli, jeśli dla obiektu, stanowiącego de facto źródło danych pojedynczego wiersza, istnieje definicja komponentu \textbf{InfoPage}. Wspomniana kolumna zawiera akcję, której wywołanie prowadzi do strony obiektu domenowego.

	\begin{code}
		\inputminted[
			linenos=true,
			firstline=35,
			lastline=77,
			fontfamily=monospace,
			obeytabs=true,
			samepage=false,
			fontsize=\scriptsize
		]{java}{\SpringatomScrPath{/webmvc/pages/builders/CarsTableBuilder.java}}
		\caption[Klasa definiująca strukturę tabeli wyświetlającej listę samochodów]{
			\emph{CarsTableBuilder} - klasa definiująca strukturę tabeli wyświetlającej listę samochodów, źródło: opracowanie własne	
		}
		\label{app:cars_table_src_code}
	\end{code}
	
	Proces, w wyniku którego tabela widoczna na rysunku \ref{app:tableInContext}, jest dostępna w interfejsie użytkownika, przedstawiony został w poniższej tabeli:
	\begin{center}
		\begin{longtable}{| p{1.5cm} | p{14cm} |}
			\caption[Proces renderowania tabeli]{Proces renderowania tabeli}\tabularnewline	
			
			% header
			\hline
				\multicolumn{1}{|c|}{\textbf{Krok}} 		&
				\multicolumn{1}{|c|}{\textbf{Opis}} 		\tabularnewline
			\hline
			\endfirsthead
			
			\multicolumn{2}{c}
			{{\bfseries \tablename\ \thetable{} -- kontynuacja...}} \tabularnewline
			\hline
				\multicolumn{1}{|c|}{\textbf{Krok}} 		&
				\multicolumn{1}{|c|}{\textbf{Opis}} 		\tabularnewline
			\hline
			\endhead
				
			\hline
				\multicolumn{2}{|r|}{{Następna strona...}} \tabularnewline
			\hline
			\endfoot
	
			\hline
			\endlastfoot	
			% end of header
			
			% body starts her
			\textbf{1}							&
			Tabela istnieje w kontekście strony domenowej. Zdefiniowany na 
			listingu \ref{app:domain_render_jsp} kod, obsługuje ten przypadek z wykorzystaniem
			technologii JavaScript. Po stronie klienta generowane jest 
			asynchroniczne zapytanie, które zostaje zmapowane na wywołanie metody \textbf{getTableBuilderPost} (listing \ref{app:svTableBuilderController})
			kontrolera \textbf{SVTableBuilderController}. Metoda odszukuje 
			instancję \textbf{TableComponentBuilder} na podstawie przekazanego w zapytaniu parametru \textbf{builderId},
			a następnie umieszcza odnaleziony obiekt w odpowiedzi razem z nazwą widoku.
			\tabularnewline
			\hline
			\textbf{2}							&
			Wybrany w punkcie 2 widok to plik JSP, w którym, z przekazanej instancji obiektu \textbf{ComponentBuilder}, pobierana jest
			definicja tabeli. Uzyskane informacje są, w dalszej części, ustawiane jako atrybuty tagu \textbf{JSP} z biblioteki \textbf{Dandelion Datables}.
			Od tego momentu, proces budowania struktury DOM nowej tabeli jest obsługiwany przez wybraną biblioteką. Warto tutaj zwrócić uwagę na atrybut \textbf{url},
			pod którym udostępnione będą dane dla tabeli. Adres tworzony jest na podstawie unikatowego identyfikatora obiektu \textbf{TableComponentBuilder}.
			\tabularnewline
			\hline
			\textbf{3}							&
			\textbf{Dandelion Datatables} po stworzeniu struktury tabeli 
			generuje asynchroniczne zapytanie pod podany w punkcie 2 adres. Żądanie
			zostaje zmapowane na wywołanie metody \textbf{getBuilderData} (listing \ref{app:svTableBuilderController}) 
			kontrolera \textbf{SVTableBuilderController}. Metoda pobiera instancję
			\textbf{ComponentBuilder} i wywołuje metodę \textbf{getData}. 
			Dane są następnie zwracane w postaci pliku \textbf{JSON} do tabeli po stronie klienta aplikacji.
		\end{longtable}
	\end{center}	
	
	\clearpage
	\begin{code}
		\inputminted[
			linenos=true,
			firstline=67,
			lastline=103,
			fontfamily=monospace,
			obeytabs=false,
			samepage=false,
			fontsize=\scriptsize
		]{java}{\SpringatomScrPath{/webmvc/controllers/SVTableBuilderController.java}}
		\caption[SVTableBuilderController - kontroler obsługujący żądania komponentu \textbf{TableBuilder}]{
			SVTableBuilderController - kontroler obsługujący żądania komponentu \textbf{TableBuilder}, źródło: opracowanie własne
		}
		\label{app:svTableBuilderController}
	\end{code}	
		
\subsection{RBuilder - szablony raportów biznesowych}
	\textbf{RBuilder} jest modułem aplikacji praktycznej, realizującym koncepcję tworzenia szablonów raportów biznesowych, będących opisem struktury gotowego raportu. Generowanie raportów zostało zrealizowane z wykorzystaniem biblioteki \textbf{Dynamic Jasper} (\ref{tech:jasperReports}). Istniejące narzędzia do tworzenia szablonów, zrozumiałych przez tę bibliotekę, nie istnieją jako aplikacje działające w środowisku przeglądarki internetowej. Głównym założeniem funkcjonalnym komponentu \textbf{RBuilder} jest więc dostarczenie logiki oraz widoku pozwalających na przygotowanie szablonu w oknie przeglądarki, a następnie wygenerowanie gotowego raportu jako dokumentu \textbf{PDF}, \textbf{XLS}, \textbf{HTML} lub \textbf{CSV}.
	
	\paragraph{Model danych - reprezentacja szablonu}   \hspace{0pt} \\
	Szablon raportu należy rozumieć jako obiekt domenowy \textbf{SReport} oraz klasę \textbf{ReportConfiguration}. Pierwszy z wymienionych obiektów zapisywany jest w bazie danych i pozwala na stwierdzenie gdzie zapisany został szablon oraz gdzie znajduje się zserializowany obiekt drugiej z wymienionych klas. \textbf{ReportConfiguration} dostarcza informacji trudnych do efektywnego zamodelowania na poziomie bazy danych. Są to informacje opisujące źródło danych:
	\begin{itemize}
		\item wybrane obiekty domenowe,
		\item wybrane atrybutu obiektów domenowych	
	\end{itemize}	 
	oraz strukturę nowego raportu, widoczną dla użytkownika:
	\begin{itemize}
		\item tytuł,
		\item podtytuł,
		\item kolumny,
		\item informacje o kolumnach grupujących, według których dane będą pogrupowane,
		\item wybrane reprezentacje, w których dane umieszczone w poszczególnych kolumnach, będą zaprezentowane w raporcie
	\end{itemize}
	
	\paragraph{Serwisy - przetwarzanie danych}  					\hspace{0pt} \\
		Serwisy komponentu \textbf{RBuilder} dostarczają metod, dzięki którym możliwe jest zapisanie szablonu, wygenerowanie z niego raportu 
	w wybranej przez użytkownika reprezentacji i dostarczenie danych do przewodnika tworzenia nowego raportu. Dostarczone usługi to:
	\begin{itemize}
		\item ustalenie możliwych formatów danego atrybuty, a tym samym kolumny w raporcie,
		\item tworzenie obiektu domenowego w zależności od konfiguracji raportu,
		\item generowanie raportu,
		\item zapis i odczyt informacji o raporcie z bazy danych oraz systemu plików.
	\end{itemize}
	Funkcjonalność tej grupy klas dla komponentu \textbf{RBuilder} można podzielić na następujące bloki:
	\begin{center}
		\begin{longtable}{| p{2.5cm} | p{13cm} |}
			\caption[Bloki funkcjonalne modelu serwisów \textbf{RBuilder}]{
				Bloki funkcjonalne modelu serwisów \textbf{RBuilder}			
			}
			\label{app:rbuilder_services_functionality_table}
			\tabularnewline	
			
			\hline
				\multicolumn{1}{|c|}{\textbf{Grupa}} &
				\multicolumn{1}{|c|}{\textbf{Funkcjonalność}} \tabularnewline
			\hline
			\endfirsthead
			
			\multicolumn{2}{c}
			{{\bfseries \tablename\ \thetable{} -- kontynuacja...}} \tabularnewline
			\hline
				\multicolumn{1}{|c|}{\textbf{Grupa}} &
				\multicolumn{1}{|c|}{\textbf{Funkcjonalność}} \tabularnewline
			\hline
			\endhead
				
			\hline
				\multicolumn{2}{|r|}{{Następna strona...}} \tabularnewline \hline
			\endfoot
			\hline
			\endlastfoot	
			
			\emph{Operation Management} 									& 
			Grupa \textbf{Operation Management} odpowiedzialna jest za tworzenie obiektu domenowego \textbf{SReport} w zależności
			od ilości tabel wybranych dla konkretnego raportu. Lista klas:
			\begin{itemize}
				\item SingleEntityRBuilderCreateOperation - obsługuje przypadek, w którym źródłem danych dla nowego raportu jest pojedyncza tabela w bazie danych,
				\item MultipleEntitiesRBuilderCreateOperation - obsługuje przypadek, w którym źródłem danych jest wiele tabel.
			\end{itemize}	
			\tabularnewline				
			\hline
			
			\emph{Data Management}											&
			Klasy z grupy \textbf{Data Management} zostały zaprojektowane do pobierania danych takich jak:
			\begin{itemize}
				\item informacje o typach obiektów domenowych, które można uwzględnić w raportach. Takie klasy adnotowane są 
				przez \emph{\@{}ReportableEntity}, a ich lista udostępniana jest poprzez interfejs \emph{ReportableEntityResolver},
				\item listę kolumn wraz z ich cechami takimi jak nazwa, typ danych przechowywanych w klasie, w odpowiadającej jej polu oraz możliwe ich reprezentacje. Informacje tego typu udostępniane są poprzez interfejs \emph{ReportableColumnResolver},
				\item listę powiązań między modelami w uproszczonej formie na potrzeby wybierania tabel podczas projektowania raportu. 
				Na obecną chwile możliwe jest utworzenie jedynie nieprzechodnich powiązań opisanych na bazowym poziomie przez relacje
				klucz główny - obcy. Dane tego typu udostępnione są przez interfejs \emph{ReportableAssociationResolver}. 
			\end{itemize}
			\hline
			
			\emph{Dynamic Jasper Operation}								&
			\emph{JasperBuilderService} jest jedyną klasą tej grupy, dostarczającą możliwości utworzenia skompilowanego
			szablonu do pliku \textbf{*.jasper}. Obecnie wspiera ona przypadek pojedynczej tabeli, jako źródła danych raportu. Jej zadaniem jest
			utworzenie obiektu klasy \textbf{DynamicReport} poprzez ustawienie, na tym obiekcie, następujących informacji:
			\begin{itemize}
				\item tytuł,
				\item podtytuł,
				\item opis,
				\item język,
				\item szerokość odpowiednich sekcji jak nagłówek, stopka itp.,
				\item lista kolumn,
				\item lista kolumn według których dane mają być grupowane.
			\end{itemize}
			\hline
			
			\emph{Generic helper}												&
			\emph{ReportBuilderService} jest artefaktem serwisu, przekazującym sterowanie do modułu
			\textbf{Operation Management}, celem utworzenia instancji obiektu domenowego \textbf{SReport} oraz wsparcia
			dla operacji renderowania raportu w konkretnej reprezentacji. Kiedy pierwsza z funkcji jest trywialna w kontekście złożoności, służąc
			jedynie separacji zadań i zmniejszeniu kohezji klas, druga z wymienionych metod jest dużo bardziej złożona.
			Jej celem jest pobranie danych wymaganych przez moduł \textbf{Spring}, używanych później do zrenderowania raportu 
			w wybranym przez użytkownika formacie, na przykład PDF. Operacje przez nią wykonywane to:
			\begin{itemize}
				\item pobranie obiektu domenowego z bazy danych dla danego numeru raportu, 
				\item deserializacja skompilowanego pliku \textit{*.jasper} z systemu plików,
				\item utworzenie źródła danych, zrozumiałego przez bibliotekę \textbf{DynamicJasper}, na podstawie informacji takich jak lista kolumn, ich typ, wybrany typ reprezentacji danych w kolumnie
			\end{itemize}
			\hline
		\end{longtable}
	\end{center}
				
	%%%%%%%%%%%%%%%%%%%%%%%%%%%%%%%%%%%%%%%%%%%%%%%%%%%%%%%%%%%%%%%%%%%%%%%%%%%%%%%%%%%%%%%%%%%%%%%%%%%%%%%%%%%%%%%%%%%%%%%%%%%%%%%%%%%%%%%%%%%%
	
	\paragraph{Warstwa widoku} 					\hspace{0pt} \\
	
	\begin{figure}[H]
		\centering
		\includegraphics[width=1.0\textwidth]{images/rbuilder_table}
		\caption[Tabela wyświetlająca zapisane szablony]{
			Tabela wyświetlająca zapisane szablony, źródło: opracowanie własne
		}
		\label{app:wizard_savedReports}
	\end{figure}	
	
	Na część obsługującą widok komponentu \textbf{RBuilder} składają się:
	\begin{itemize}
		\item tabela z istniejącymi raportami,
		\item przewodnik tworzenia nowego raportu (opisany w rozdziale \textbf{ReportBuilder - tworzenie szablonów raportów biznesowych} \ref{wizard:rbuilder})
	\end{itemize}

	Tabela jest obiektem należącym do warstwy widoku, utworzonym z wykorzystaniem komponentu \textbf{TableBuilder} (\ref{app:tableComponentBuilder}). W tabeli
	widoczne są wszystkie zapisane w systemie szablony. Dodatkowo do każdego z wierszy przypisane zostały akcje pozwalające na:
	\begin{itemize}
		\item generowanie nowego raportu,
		\item usunięcie szablonu z bazy danych.
	\end{itemize}
	
	\begin{figure}[H]
		\centering
		\includegraphics[width=0.5\textwidth]{images/rbuilder_generateReport}
		\caption[Generowanie nowego raportu - wybór docelowego formatu]{
			Generowanie nowego raportu - wybór docelowego formatu, źródło: opracowanie własne
		}
		\label{app:wizard_newReport_generateReport}
	\end{figure}
	
	Usunięcie jest prostą, z punktu widzenia użytkownika, operacją, ponieważ nie widzi on niczego poza końcowym rezultatem. Z drugiej strony w momencie kliknięcia na przycisk \textbf{Generuj}, użytkownik ma możliwość wybrania końcowego formatu, w jakim chciałby zobaczyć swoje dane. Przykładowy raport wygląda w następujący sposób:
	
	\begin{figure}[H]
		\centering
		\includegraphics[width=1.0\textwidth]{images/rbuilder_report}
		\caption[Gotowy raport utworzony przez komponent \textbf{RBuilder}]{
			Gotowy raport utworzony przez komponent \textbf{RBuilder}, źródło: opracowanie własne
		}
		\label{app:wizard_newReport_report}
	\end{figure}	
	

\section{Plany rozwojowe}								Aplikacja demonstracyjna jest obecnie na etapie dalszego rozwoju. Posiada szeroko zdefiniowane moduły zaprojektowane, aby wspierać takie
rejony jak:
\begin{itemize}
	\item generyczna warstwa operacji bazodanowych,
	\item warstwa logiki biznesowej udostępnionej przez serwisy,
	\item moduł komponentów dla budowy tabel oraz stron obiektów domenowych,
	\item selektywna warstwa konwersji,
	\item tagi oddelegowane dla przewodników opartych o Spring Web Flow,
	\item model danych wraz z jego abstrakcyjną warstwą (interfejsy) do użytku zewnętrznego,
	\item moduł obsługujący funkcjonalność raportowania.
\end{itemize}

W większości przypadków uzyskana funkcjonalność jest jednak na etapie implementacji i niektóre z niedopracowanych elementów 
ulegną zmianie, celem uproszczenia zarządzania złożonością projektu oraz usunięcia nadmiarowych klas, jak także słabo konfigurowalnych części.
Plany rozwojowe aplikacji zostały przedstawione w tabeli \ref{app:changes_in_future}.
\begin{center}
	\begin{longtable}{|l|p{10cm}|}
		\caption[Plany rozwojowe aplikacji demonstracyjnej]{
			Plany rozwojowe aplikacji demonstracyjnej
		}
		\label{app:changes_in_future}
		\tabularnewline	
		
		\hline
			\multicolumn{1}{|c|}{\textbf{Moduł}}		&
			\multicolumn{1}{|c|}{\textbf{Opis zmian}}	\tabularnewline
		\hline
		\endfirsthead
		
		\multicolumn{2}{c}
		{{\bfseries \tablename\ \thetable{} -- kontynuacja...}} \tabularnewline
		\hline
			\multicolumn{1}{|c|}{\textbf{Moduł}}		&
			\multicolumn{1}{|c|}{\textbf{Opis zmian}}	\tabularnewline
		\hline
		\endhead
			
		\hline
			\multicolumn{2}{|r|}{{Następna strona...}} \tabularnewline \hline
		\endfoot
		\hline
		\endlastfoot	
		
		\textbf{ComponentBuilder}	&
		Ogólne rozszerzenie i optymalizacja modułu \textbf{ComponentBuilder} dotyczyć będzie:
		\begin{itemize}
			\item wsparcie dla cache'owania raz załadowanych komponentów,
			\item przeniesienie definicji stron do deklaratywnego języka XML.
			Będzie to możliwe po usprawnieniu działania selektywnych konwerterów oraz 
			zwiększy możliwość zmian zgodnie z wymaganiami, bez konieczności zmian w plikach Java,
			\item przeniesienie kodu widoku stron domenowych do łatwiejszego w zarządzaniu oraz utrzymaniu kodu ExtJS,
			\item wprowadzenie automatycznego mechanizmu generującego przekierowania do stron obiektów domenowych,
			\item połączenie kontrolerów odpowiedzialnych za obsługę żądań pochodzących od komponentów typu \textbf{tabele} oraz \textbf{strony domenowe}. Celem tego działania jest ujednolicenie adresów sugerujące, że oba komponenty służą podobnemu celowi.
		\end{itemize}
		\hline		
		\textbf{WebFlow}			&
		Zaprojektowanie biblioteki wspierającej 
		funkcjonalność \textbf{Spring Web Flow} dla biblioteki ExtJS. Decyzja podyktowana jest 
		chęcią zminimalizowania użycia różnorodnych bibliotek JavaScript. Gotowa biblioteka
		byłaby udostępniona na licencji OpenSource.
		\hline
		\textbf{Selektywne konwertery}	&
		Rozszerzenie możliwości selektywnych konwerterów o obiekty inne niż domenowe oraz
		wsparcie dla możliwości definiowana powtarzających się selektorów - kluczy w kontekście
		globalnym, ale unikatowych w kontekście danego obiektu podlegającego konwersji.
		\hline
		\textbf{Kod ogólny}				&
		Usunięcie pozostałości nadmiarowego kodu, który pozostał po uaktualnieniu aplikacji,
		do korzystania z najnowszej wersji (4.0.0) szkieletu aplikacji \textbf{Spring}.
		\hline
		\textbf{Widok}					&
		Wsparcia dla ExtJS - migracja istniejących komponentów warstwy widoku użytkownika do 
		szkieletu aplikacji ExtJS. Prawdopodobnym wyborem będzie wykorzystanie ExtJS w wersji 4.x.x z uwagi
		na lepszą wydajność i większe możliwości biblioteki, w porównaniu do poprzednich wersji.
		\hline
		\textbf{RBuilder}			&
		Zrezygnowania z ciężkiego w użyciu oraz utrzymaniu sposobu generowania raportów w aplikacji poprzez
		bibliotekę \textbf{DynamicJasper}. Przeniesienie bezpośredniego generowania raportów do tabelek oraz
		inwestygacja możliwych technologii do wykorzystania w przypadku eksportowania raportu do formatów takich jak
		HTML, CSV, PDF oraz XLS.		
		\hline
		\textbf{Przewodniki}			&
		Ukończenie wszystkich wymaganych kreatorów służących zarówno do modyfikacji, jak i tworzenia nowych obiektów domenowych.
		\hline
		\textbf{Terminarz spotkań}		&
		Ukończenie prac nad terminarzem spotkań. Obecna funkcjonalność pozwala na tworzenie obiektów, które następnie
		można przeglądać z użyciem kalendarza (podobnego do używanego w programie Outlook). 	
		\hline
		\textbf{Dekodowane numeru VIN}		&
		Rozszerzenie obecnych możliwości dekodera numeru VIN o pobieranie informacji, takich jak:
		\begin{itemize}
			\item wytwórca samochodu,
			\item marka oraz model,
			\item typ nadwozia,
			\item typ paliwa,
			\item typ silnika
		\end{itemize}							
	\end{longtable}
\end{center}