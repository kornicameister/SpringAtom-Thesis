\chapter{Szkielety aplikacji - szkielety programistyczne}

\section{Rozwój szkieletów aplikacji dla WWW}
	\subsection{Podwaliny szkieletów aplikacji}
		Od momentu wynalezienia sieci internetowej znanej dzisiaj pod nazwą \textbf{WWW}\footnote{WWW - World Wide Web jest to system dokumentów połączonych ze sobą hiperłączami 
		dostępnymi przez sieć internetową} powstało, w i dalszym ciągu powstaje, wiele narzędzi pozwalających na projektowanie i tworzenie stron. Jest to jednocześnie przykład obrazujący
		prostą ideę będącą zalążkiem wielu późniejszych wynalazków. Początkowo statyczne treści, a dokładniej sposoby ich dostarczania-programy były pisane w językach takich jak C oraz Perl.
		Okazało się to jednak niewystarczające ponieważ obecnie istnieje wiele rozwiązań dla wielu różnych języków, z których warto przytoczyć między innymi Django dla języka Python, ASP.NET\cite{asp_net} dla języka C#, Zend.
		Framework dla języka PHP, ExtJS dla języka JavaScript, czy też ostatecznie Spring Framework dla języka Java. Motorem napędzającym rozwój tych narzędzi 
		były niewystarczające interfejsy programistyczne oraz rosnące oczekiwania odbiorców w stosunku do serwowania treści dynamicznych, a nie statycznych\cite{art_of_java_web_dev}.
		\todo[inline, color=red!40]{books.bib}
	\subsection{Historia i rozwój}
		Początek istnienie szkieletów aplikacji lub bardziej ogólnie szkieletów programistycznych sięga momentu kiedy pojawiły się pierwsze faktyczne aplikacje internetowe.
		Z drugiej strony sam język Java został zaprojektowany nie z myślą tworzenia rozbudowanych programów będących źródłem logiki dla stron internetowych. 
		Pierwsza koncepcja tego języka jako sposobu tworzenia treści Internetu pojawiła się dopiero w z ideą \textit{serwletów} w roku 1995, 
		jeszcze przed wydaniem pierwszej wersji języka Java. Pierwsza działająca wersja \textbf{serwletów} pojawiła się z rokiem 1997, a jej twórcami
		byli pomysłodawcy James Gosling oraz Pavani Diwanji. Szybko okazało się jednak, że ta początkowo obiecująca idea jest niewystarczająca.
		Różnice paradygmatów związane z projektowaniem kodu warstwy widoku oraz kodu Javy oferującego funkcjonalność były zbyt duże. 
		Długotrwała utrzymanie oraz modyfikacje istniejącego kodu okazywały się na dłuższą metę niemożliwe lub bardzo trudne\cite{art_of_java_web_dev}. 
		
		Kolejnym krokiem były wprowadzenie szablonów, które w znaczący sposób usprawniły pracę programistów odpowiedzialnych za wizualną stronę
		aplikacji oraz tych odpowiedzialnych za generowanie dynamicznej jej treści. Mimo wszystko rozwiązanie to miało jedną znaczącą wadę - wydajność. 
		Początkowo implementacje opierały się na każdorazowym przeszukania kodu HTML w poszukiwaniu odpowiednich znaczników i zastępowaniu ich odpowiadającymi
		im wartościami. Takie podejście przestałoby być efektywnym z momentem kiedy z witryny zaczynało korzystać więcej niż 1 osoba, ponieważ dla każdego
		z kolejnych odwiedzających należało wykonać ten sam zestaw operacji aby pokazać gotową stronę. Parsowanie plików HTML zawierających znaczników
		generowało zbyt duże obciążenie zarówno w kontekście zużycia zasobów procesora jak i operacji I/O\footnote{Operacje dyskowe, na plikach, 
		często określane mianem operacji Wejścia/Wyjścia}\cite{art_of_java_web_dev}. 
		\begin{code}
			\inputminted[
				lineos=true,
				firstline=23,
				lastline=40,
				fontfamily=monospace,
				obeytabs=true,
				samepage=true,
				fontsize=\scriptsize
			]{jsp}{../SpringAtom/src/main/webapp/ui/wizard/header.jsp}
			% src file used
			\label{jsp_tag_usage_example}
			\caption[Przykład użycia tagów JSP]{\textit{header.jsp} - plik JSP definiujący nagłówek \textbf{wizard} wykorzystujący język znaczników do tworzenia dynamicznej treści}.
		\end{code}
		
	 	Odpowiedzią oraz jednocześnie kolejnym krokiem ku szkieletom programistycznym stały się \textbf{JavaServer Pages}\cite{jsp_spec}. 
		\textbf{JSP} dalej korzystały i wciąż korzystają z szablonów ale podejście jest zupełnie inne. 
		Zamiast parsować stronę za każdym razem w poszukiwaniu znaczników, \textbf{JSP} są parsowane tylko na początku i następnie konwertowane
		do serwletów, a następnie wykonywane. Każde następnie odwołanie do tego same adresu nie powoduje ponownej kompilacji strony \textbf{JSP} do serwletu,
		ponieważ wykorzystywane jest ten skompilowany przy poprzednim odwołaniu. Niemniej korzystanie z takiego rozwiązania, mimo że rozwiązuje problemy
		związane z wydajnością, rodzi inne. Idea której głównym założeniem było pisanie plików HTML zawierających odpowiednie instrukcji do generowanie 
		treści dynamicznych przerodziło się niekontrolowanie w trudny do utrzymania i zrozumienia kod, gdzie wymieszane były elementy odpowiedzialne 
		za logikę biznesową oraz warstwę widoku\cite{art_of_java_web_dev}.
	 	\begin{code}
			\inputminted[
				lineos=true,
				fontfamily=monospace,
				obeytabs=true,
				samepage=true,
				fontsize=\scriptsize
			]{jsp}{codeSamples/bad_jsp.jsp}
			\caption[Przykład wymieszania logiki biznesowej oraz warstwy widoku w pliku JSP]{Przykład wymieszania logiki biznesowej oraz warstwy widoku w pliku JSP}.
			\label{bad_jsp_usage_example}
		\end{code}
		Listing \ref{bad_jsp_usage_example} obrazuje takie nadużycie. W poprawnym przykładzie pobranie listy pustych jednostek odbyłoby się poza plikiem JSP. 
		
\section{Wzorce projektowe jako budulec szkieletów aplikacji}
	Kolejnym krokiem w historii języka \textbf{Java} jako narzędzie do budowy stron internetowych okazały się wzorce projektowe.
	Będąc odpowiedzią na problemy wynikające z niepoprawnego korzystania z plików \textbf{JSP} stały się jednocześnie punktem
	pośredniczącym, gdzie następnym etapem były właściwe szkielety aplikacji jako praktyczne przykłady ich wykorzystania.
	
	Za twórcę pojęcia \textbf{wzorzec projektowy} nie uznaje się żadnego programisty, ale wkładu Christopher'a Alexender'a nie sposób pominąć.
	Zauważył on, podczas swoich licznych podróży biznesowych, że architekci z wielu różnych krajów ergo całkowicie ze sobą niezwiązani, 
	przejawią tendencję do rozwiązywanie podobnych klas problemów w równie podobny sposób. Owocem tego spostrzeżenia było wydanie 
	książki traktujących o najlepszych praktykach dla osób jemu podobnych\cite{christoper_alexander_influence}. Mimo, że odnosił się on 
	do problemów całkowicie niezwiązanych z programowaniem, jego idea została po raz pierwszy zaadoptowana dla kwestii programistycznych 
	w roku 1994 przez Erich'a Gamma, Richard'a Helm, Ralph'a Johnson oraz John'a Vlissides w książce \textit{Design Patterns: Elements of Reusable Object-Oriented Software}. 
	Zdefiniowane w niej pojęcia wzorca stało się samo w sobie wzorem do tworzenia kolejnych. 

\section{Znaczenie szkieletów aplikacji}
	Internetowe szkielety aplikacji nie są same w sobie bibliotekami programistycznymi. Stanowią one raczej ich zbiór oraz nierzadko i 
	narzędzi mających na celu ułatwienie programiście implementacji własnego rozwiązanie. Bardzo często są one również praktyczną 
	implementacją standardów i tzw. \textbf{best practices}\footnote{Zalecane i pożądane sposoby realizacji często spotykanych problemów}. 
	Jest to szczególnie użyteczne ponieważ nierzadko zdarza się, że programista popełnia błąd na pewnym etapie projektowanie lub implementacji 
	pewnego modułu, którego późniejsze konsekwencje wymagają stworzenie niepotrzebnego kodu, czego dałoby by się uniknąć, gdyby podążano
	już wyznaczonymi ścieżkami. Prawdziwe w takim wypadku staje się również zdanie, że jeden błąd generuje kolejne, a te mogą być zalążkiem następnych.
	To czym są szkielety aplikacji u samego źródła ich istnienia jest zapobiegać takim sytuacjom poprzez proponowanie już gotowych modułów,
	które są przetestowane i ciągle modyfikowane przez doświadczone osoby celem dostarczenia jeszcze lepszych rozwiązań\cite{art_of_java_web_dev}.
	
	Oprogramowanie zorientowane obiektowo jest doskonałym zobrazowaniem koncepcji wykorzystanie szkieletu jako fundamentu do budowy własnego rozwiązania.
	Na najniższym poziomie szczegółowości każdy program czy też moduł większej części jest zbiorem klas posiadającym jasno określony zbiór ról - obowiązków, a 
	których obiekty współpracują ze sobą celem dostarczenie gotowego wyniku lub jego części. Wspólnie te obiekty reprezentują pewną koncepcję, dla której został utworzone.
	W kontekście framework'a dla aplikacji internetowych można więc wyróżnić klasy przeznaczone dla kooperacji z bazą danych, odpowiedzialnych za walidacji informacji czy też 
	pomocnych w momencie renderowania widoku. 
	Warto nadmienić, że te zasady są równie ważne dla małych systemów, jak i dla dużych. Niemniej w pierwszym przypadku, gdzie poziom skomplikowania jest niski nie ma 
	potrzeby definiowania wielu poziomów abstrakcji ułatwiających określone czynności, jak na przykład wcześniej wymienione walidacje danych. Niestety z czasem, początkowo
	prosty system, staje się coraz bardziej skomplikowany i bardzo często programista nie jest już wtedy w stanie zapanować na chaosem oraz dostarczyć zunifikowanego sposobu
	rozwiązywania powtarzalnych czynności. 
	Z tego powodu dobry framework charakteryzuje się jasno, ale nie sztywno, zdefiniowanymi granicami między poszczególnymi zbiorami funkcjonalnymi. Wprowadzone poziomy
	abstrakcji, często więcej niż jeden dla pojedynczego celu jak na przykład sposób interakcji systemu i jego klientów, są wynikiem wieloletnich zmian podczas kiedy zidentyfikowano
	wiele wspólnych problemów i dla których znaleziono rozwiązanie w postaci ram projektowych czy też \textbf{best pratices}, będących ostatecznie właściwą 
	esencją znaczenie szkieletu aplikacji\cite{framework_design_-_a_role_modeling_approach}.
	
	Dobrymi przykładami tutaj będą z pewnością warstwy abstrakcji dla obsługi operacji bazodanowych. Zawierając konkretne implementacje,
	które już posiadają funkcjonalność odpowiedzialną za wykonanie tychże operacji na praktycznie elementarnym poziomie, 
	a zostawiając właściwą warstwę logikę w kontekście tworzonej aplikacji, odciążają one programistę od przysłowiowego wynajdowania koła od nowa.
	Praktyczną realizacją tej koncepcji jest na przykład \textbf{Spring Data}, które pozwala na napisania kodu, którego głównymi zaletami
	będzie odseparowanie logiki biznesowej od wybranej bazy danych oraz wyraźny podział na klasy odpowiedzialne za operacje \textbf{CRUD} 
	na danych, jak i te wykonujące operacje biznesowe. Inne przykłady to między innymi \textbf{EJB}\todo[inline,color=red]{books.bib} 
	czy też moduł innego szkieletu programistycznego \textbf{GWT}\todo[inline,color=red]{books.bib} wykonującego identyczne zadanie. 
	Warto nadmienić, że również warstwy odpowiedzialne za tworzenie i zarządzanie widokiem (warstwa prezentacji),
	czy też takie których nadrzędnym celem jest pośredniczenie między widokiem, a danymi są potencjalnymi 
	kandydatami do wyodrębnienie pewnego zbioru funkcjonalności jako części składowych gotowe szkieletu aplikacji. 
	
	\subsection{Funkcjonalność szkieletów aplikacji}
	
	Do najczęściej implementowanych funkcjonalności framewerków webowych należą:
	\begin{itemize}
		\item internacjonalizacja oraz lokalizacja tworzonych stron w dowolnym języku oraz wsparcia dla efektywnego przełączanie między nimi
		\item wsparcia dla technologi widoku innych niż strony \textbf{JSP} lub takich, które je wykorzystują ale definiują inny sposób korzystania z nich
		\item integracja z językiem szablonów innym niż \textbf{JSTL}
		\item walidacja danych
		\item mapowanie żądań HTTP do tzw. \textbf{kontrolerów}\footnote{Kontrolery należy rozumieć jako tzw. \textbf{POJO} które same w sobie są zwykłymi klasami, ale w kontekście framework'a nabierają konkretne znaczenie jako wykonawcy pewnej logiki właściwej dla danej aplikacji}
		\item wsparcie dla popularnych języków transportu danych jak JSON czy też jego odpowiednik XML. 
	\end{itemize}
	
	\subsection{Problemy szkieletów aplikacji}
	Mimo że framework'i znacząco podnoszą jakość kodu aplikacji oraz obniżają późniejsze koszty jej utrzymania nie są doskonałym
	narzędziem. Większość niedociągnięć, które można zaobserwować związane jest z:
	\begin{itemize}
		\item \textit{złożonością klas} - obiekty klas zaimplementowane w szkielecie aplikacji współpracują ze sobą, wielokrotnie w
		więcej niż jednym kontekście. Definiowania funkcjonalności danej klasy poprzez użycie pojedynczej klasy abstrakcyjnej lub
		interfejsu jest rozwiązanie zbyt sztywnym ponieważ często większa część zdefiniowanych metod nie będzie wykorzystana 
		w innym miejscu,
		\item \textit{skupieniem się na szczególe} - w momencie projektowania klas, tj. kreowania późniejszego celu istnienia 
		ich obiektów, zdarza się, że gubi się obraz całości zbytnio skupiając się na poszczególnych przypadkach, 
		\item \textit{złożoność współpracy} - mechanizmy współpracy obiektów odpowiadających za, na przykład, komunikacją klient-serwer
		mogą stać się zbyt skomplikowane,
		\item \textit{trudnością użycia} - brak drobiazgowej dokumentacji może skutkować użyciem szkieletu w sposób niezamierzony przez
		jego tworców, co może skutkować implementowaniem tzw. \textbf{work arounds}\footnote{Technika programistyczna, której celem jest
		naprawa jakiegoś błędu bądź uzyskanie zamierzonego celu podczas gdy używany framework lub biblioteka nie pozwalają na dotarcie
		doń.} lub błędami funkcjonalnymi\cite{framework_design_-_a_role_modeling_approach}. 
	\end{itemize}
	
\section{Istniejące architektury szkieletów - realizacje wzorców programistycznych}
	