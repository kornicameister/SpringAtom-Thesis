\chapter{Wstęp}
\label{chaper:introduction}

\section{Uzasadnienie i wybór tematu}
	Od kilku lat obserwuje się wzrost znaczenie stron internetowych jako narzędzi pracy dla różnorodnych firm czy też przedsiębiorstw. Firmy produkcyjne, dystrybucyjne a także i sklepy korzystają z takich rozwiązań z uwagi na szybkość wymiany informacji i ich globalnego zasięgu, a także braku konieczności instalacji dodatkowych bibliotek jest niezaprzeczalną zaletą. Nie można w tym miejscu zapomnieć także o korzyściach finansowych wynikających ze spójności sposobu dostarczenia. 
	Obecnie jednym z najpopularniejszych zastosowań języka Java jest tworzenie komercyjnych aplikacji internetowych. Nie wystarczające są rozwiązanie znane z początków tego języka w tym kontekście, z uwagi na wciąż rosnące potrzeby i wymagania klientów. Rezultatem tego były narodziny nowej gałęzi programów, a dokładniej ich zbioru - szkieletów aplikacji programistycznych, których głównym celem jest odciążenie programisty oraz wzrost jakości dostarczanych rozwiązań. Popyt generuje podaż - to zdanie jest wciąż bardzo aktualne ponieważ potrzeba lepszej jakości rozwiązań była motorem napędowym ich tworzenia i obecnie na rynku istnieje ponad 50 framework'ów otwarto źródłowych napisanych w języku Java i przeznaczonych dla pisania aplikacji JavaEE. 
	Wybrany temat obrazuje użycie jednego ze szkieletów aplikacji dla przygotowania kompleksowego rozwiązania wspierającego warsztat samochodowy. 
\section{Cel i zakres pracy}
		Głównym celem pracy jest przygotowanie aplikacji internetowej z wykorzystaniem szkieletu aplikacji Spring, którego głównym zadaniem
	jest globalne wsparcia pracy przedsiębiorstwa zajmującego się naprawą oraz serwisowaniem samochodów, zarówno osobowych jak i ciężarowych.
	Szczególna uwaga zostanie poświęcona zagadnieniom związanym z tworzeniem efektywnego modelu danych oraz bezpieczeństwu wrażliwych danych takie jak dane osobowe, numery rejestracyjne oraz VIN pojazdów. Szczególna uwaga zostanie poświęcona rozpoznaniu i zaimplementowaniu mechanizmu wspierającego trzymanie
	historii zmian dla zagadnień takich jak historia własności pojazdu czy też zmian w profilach użytkowników.
		Z drugiej aplikacji ma dostarczyć wiedzy o budowania aplikacji biznesowej w języku Java przy wykorzystaniu Spring'a. Z tego powodu zagadnieniom takim jak:
	\begin{itemize}
		\item \textit{decoupling} - minimalizacja zależności poszczególnych modułów od siebie,
		\item \textit{inteface based programming} - ukrywanie właściwych implementacji modułów, których funkcjonalność jest dostępna przez publiczne interfejsy,
		\item uproszczenie konfiguracji aplikacji - poprzez zastosowania centralnych plików konfiguracyjnych wspartych plikami \textbf{*.properties} zawierającymi
		właściwe ustawienie, co przekładać się ma na zwiększenie konfigurowalności aplikacji,
		\item walidacją danych sprowadzającą się do właściwej konfiguracji aplikacji oraz dostarczeniu wymaganych adnotacji,
		\item interakcją z użytkownikiem poprzez więcej niż tylko interfejs graficzny, ale także elementy takie jak wysyłania wiadomości oraz okresowe wykonywania określonych
		zadań,
		\item zminimalizowanie ilości koniecznego kodu do przygotowania działającej warstwy danych (repozytoria, serwisy),
		\item zabezpieczeniu określonych elementów interfejsu użytkownika przez nieuprawnionym dostępem
	\end{itemize}
		
