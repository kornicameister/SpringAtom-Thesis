\chapter{Wstęp}
\label{chaper:introduction}

\section{Uzasadnienie i wybór tematu}
	
	Wzrost znaczenia aplikacji internetowych jako narzędzi pracy dla różnorodnych firm czy też przedsiębiorstw wciąż rośnie. Firmy produkcyjne, dystrybucyjne oraz sklepy korzystają z tych rozwiązań z uwagi na szybkość wymiany informacji oraz jej globalny zasięg. Ponadto brak konieczności instalacji dodatkowych bibliotek również przemawia za wyborem takiej, a nie innej formy wspomagania działalności. Również korzyści finansowe wynikające z uproszczonego modelu dystrybucji oprogramowania, oszczędność czasu i wzrost efektywności stanowią o sile aplikacji internetowych.
	
	Obecnie jednym z najpopularniejszych zastosowań języka Java jest tworzenie komercyjnych aplikacji internetowych. Nie wystarczające są już rozwiązania znane z początków istnienia tego języka, z uwagi na wciąż rosnące potrzeby i wymagania klientów. Rezultatem tego były narodziny nowej gałęzi programów, a dokładniej ich zbioru - szkieletów aplikacji programistycznych, których głównym celem jest odciążenie programisty oraz wzrost jakości dostarczanych rozwiązań. Popyt generuje podaż - to zdanie jest wciąż bardzo aktualne ponieważ potrzeba lepszej jakości rozwiązań była motorem napędowym ich tworzenia i obecnie na rynku istnieje ponad 50 szkieletów aplikacji, z publicznie dostępnym kodem źródłowym oraz dobrej jakości dokumentacją techniczną, napisanych w języku Java i przeznaczonych dla aplikacji biznesowych. 
	
	Obecnie na rynku istnieje co najmniej kilkanaście rozwiązań dedykowanych w mniejszym lub większym stopniu dla warsztatów samochodowych:
	\begin{itemize}
		\item \href{http://www.netsystem.info.pl/produkt/20}{\textbf{Warsztat NS}},
		\item \href{http://www.warsztat24.com/warsztat24-funkcjonalnosci}{\textbf{Warsztat 24}},
		\item \href{http://www.algorytm.pl/}{\textbf{Warsztat 3.4}},
		\item \href{http://meteoryt.pl/Asystent_Warsztat-k240.html}{\textbf{Asystent Warsztat}}
	\end{itemize}		
	
	Wspólnym mianownikiem dla wyżej wymienionych aplikacji jest zestaw funkcji przez nich realizowanych. Planowanie pracy warsztatu: data, godzina, czas realizacji, przypisany mechanik czy też lista czynności do wykonania przy pojeździe klienta, historia zleceń, kartoteka klientów, samochodów to jedynie niewielka część z nich. Niemniej wszystkie znalezione aplikacje łączy również brak otwartego kodu źródłowego oraz brak w pełni funkcjonalnych wersji dostępnych bez opłat. Część ze znalezionych rozwiązań to programy instalowane na komputerach docelowych, nie obsługiwane poprzez interfejs przeglądarki internetowej.  
	
	Wybrany temat obrazuje użycie jednego ze szkieletów aplikacji dla przygotowania kompleksowego rozwiązania wspierającego warsztat samochodowy, które rozwiązywałoby te problemy. Program, będący częścią praktyczną pracy dyplomowej, będzie aplikacją dostępną przez przeglądarkę internetową, bez konieczności instalacji dodatkowych bibliotek lub aplikacji, z otwartymi źródłami oraz dostępną bez opłat. 

\section{Cel i zakres pracy}
	Głównym celem pracy jest poznanie wybranego szkieletu aplikacji internetowych, zrozumienie zasad pracy z nim w celu przygotowania aplikacji internetowej wspierającej misję przedsiębiorstwa - warsztatu samochodowego. 
	
	Pod kątem funkcjonalności, szczególna uwaga zostanie poświęcona zagadnieniom związanych z tworzeniem nowych wizyt, rejestracją klientów, pojazdów i mechaników oraz utrzymaniem ich kartoteki oraz skomplikowanego modelu danych. Kompleksowy model danych będzie musiał dostarczać wszelkich możliwych informacji odnośnie obiektów, które opisuje. Przykładowo, kluczowymi atrybutami opisującymi samochód, nie będą jedynie \textbf{numery rejestracyjny}, \textbf{marka}, czy też \textbf{model}. Ważne stanie się opisania danego pojazdu parametrami takimi jak: \textbf{numer VIN}, \textbf{rok produkcji}, \textbf{spalane paliwo} oraz \textbf{parametry silnika}. Ostatecznie, wzięte pod uwagę zostanie utrzymanie historii własności pojazdu. Ponadto dla niektórych obiektów modelu danych zostanie dostarczone przeglądanie historii zmian ich atrybutów. 
	
	Drugim celem pracy jest dostarczyć wiedzy o projektowaniu oraz implementacji aplikacji biznesowych w języku Java przy wykorzystaniu Spring'a. Szczególnie ważny stanie się tutaj aspekt samego projektowania aplikacji, realizowany poprzez wykorzystanie standardowych, gotowych rozwiązań dostarczonych przez wybrany szkielet aplikacji. Mowa tutaj o takich elementach jak praca z danymi (operacje zapisu i odczytu, walidacja), separacja modelu danych i logiki biznesowej, efektywne mapowanie żądań klienta do serwera.
	Ponadto na potrzeby aplikacji demonstracyjnej zaprojektowane i zaimplementowane zostaną generyczne moduły wspierające wyświetlanie informacji o obiektach modelu danych, tabeli oraz tworzenia dynamicznych szablonów dla raportów biznesowych. Wspomniane moduły będą miały za zadanie wesprzeć dalszy rozwój aplikacji poprzez uproszczenie i automatyzację następujących przypadków: wprowadzenie nowego typu danych i dodanie dla niego stron wyświetlającej jego atrybuty, utworzenie nowej tabeli (zarówno dla nowego typu danych, ale także wyświetlającej informacje niezwiązane z modelem biznesowym), modyfikacja lub tworzenie nowego szablonu dla raportu. 