\chapter{Wstęp}
\label{chaper:introduction}

\section{Uzasadnienie i wybór tematu}
	
	Wzrost znaczenia aplikacji internetowych jako narzędzi pracy dla różnorodnych firm czy też przedsiębiorstw wciąż rośnie. Firmy produkcyjne, dystrybucyjne oraz sklepy korzystają z tych rozwiązań z uwagi na szybkość wymiany informacji oraz jej globalny zasięg. Ponadto brak konieczności instalacji dodatkowych bibliotek również przemawia za wyborem takiej, a nie innej formy wspomagania działalności. Również korzyści finansowe wynikające z uproszczonego modelu dystrybucji oprogramowania, oszczędność czasu i wzrost efektywności stanowią o sile aplikacji internetowych.
	
	Obecnie jednym z najpopularniejszych zastosowań języka Java jest tworzenie komercyjnych aplikacji internetowych. Nie wystarczające są już rozwiązania znane z początków istnienia tego języka, z uwagi na wciąż rosnące potrzeby i wymagania klientów. Rezultatem tego były narodziny nowej gałęzi programów, a dokładniej ich zbioru - szkieletów aplikacji programistycznych, których głównym celem jest odciążenie programisty oraz wzrost jakości dostarczanych rozwiązań. Popyt generuje podaż - to zdanie jest wciąż bardzo aktualne ponieważ potrzeba lepszej jakości rozwiązań była motorem napędowym ich tworzenia i obecnie na rynku istnieje ponad 50 framework'ów, z publicznie dostępnym kodem źródłowym oraz dobrej jakości dokumentacją techniczną, napisanych w języku Java i przeznaczonych dla aplikacji biznesowych. 
	
	Obecnie na rynku istnieje co najmniej kilkanaście rozwiązań dedykowanych w mniejszym lub większym stopniu dla warsztatów samochodowych. Wspólnym mianownikiem dla wyżej wymienionych aplikacji jest zestaw funkcji przez nich realizowanych. Planowanie pracy warsztatu: data, godzina, czas realizacji, przypisany mechanik czy też lista czynności do wykonania przy pojeździe klienta, historia zleceń, kartoteka klientów, samochodów to jedynie niewielka część z nich. Niemniej wszystkie znalezione aplikacje łączy również brak otwartego kodu źródłowego oraz brak w pełni funkcjonalnych wersji dostępnych bez opłat. Część z nich to także programy działające na komputerze użytkownika, a nie w przeglądarce internetowej.
	\todo[inline, color=yellow!40]{Czy potrzeba wypisać programy które znalazłem ?}
	
	Wybrany temat obrazuje użycie jednego ze szkieletów aplikacji dla przygotowania kompleksowego rozwiązania wspierającego warsztat samochodowy, które rozwiązywałoby te problemy. Program, będący częścią praktyczną pracy dyplomowej, będzie aplikacją dostępną przez przeglądarkę internetową, bez konieczności instalacji dodatkowych bibliotek lub aplikacji, z otwartymi źródłami oraz dostępną bez opłat. 

\section{Cel i zakres pracy}
	Głównym celem pracy jest poznanie wybranego szkieletu aplikacji internetowych, zrozumienie zasad pracy z nim w celu przygotowania aplikacji internetowej wspierającej misję przedsiębiorstwa - warsztatu samochodowego. 
	
	Pod kątem funkcjonalności, szczególna uwaga zostanie poświęcona zagadnieniom związanym z tworzeniem efektywnego modelu danych oraz bezpieczeństwu wrażliwych informacji takich jak dane osobowe, numery rejestracyjne oraz numery VIN pojazdów, utrzymanie i dostęp do kartoteki pojazdów, mechaników oraz klientów. 
	
	Drugim celem pracy jest dostarczyć wiedzy o projektowaniu oraz implementacji aplikacji biznesowych w języku Java przy wykorzystaniu Spring'a. Z tego powodu zagadnieniom takim jak: minimalizacja ilości kodu realizującego warstwę danych oraz ich walidacja, skupienie się na właściwej funkcjonalności aplikacji, przygotowanie generycznych modułów upraszczających dalszy rozwój aplikacji, jej utrzymanie oraz elastyczny model konfiguracji poświęcona zostanie szczególna uwaga. Ponadto przygotowany program ma służyć jednocześnie stworzeniu kilku potencjalnych modułów, które można by potem wykorzystać w innych projektach do podobnych celów.
	
	
		
