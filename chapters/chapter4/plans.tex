Aplikacja demonstracyjna jest obecnie na etapie dalszego rozwoju. Posiada szeroko zdefiniowane moduły zaprojektowane, aby wspierać takie
rejony jak:
\begin{itemize}
	\item generyczna warstwa operacji bazodanowych,
	\item warstwa logiki biznesowej udostępnionej przez serwisy,
	\item moduł komponentów dla budowy tabel oraz stron obiektów domenowych,
	\item selektywna warstwa konwersji,
	\item tagi oddelegowane dla przewodników opartych o Spring Web Flow,
	\item model danych wraz z jego abstrakcyjną warstwą (interfejsy) do użytku zewnętrznego,
	\item moduł obsługujący funkcjonalność raportowania.
\end{itemize}

W większości przypadków uzyskana funkcjonalność jest jednak na etapie implementacji i niektóre z niedopracowanych elementów 
ulegną zmianie, celem uproszczenia zarządzania złożonością projektu oraz usunięcia nadmiarowych klas, jak także słabo konfigurowalnych części.
Plany rozwojowe aplikacji zostały przedstawione w tabeli \ref{app:changes_in_future}.
\begin{center}
	\begin{longtable}{|l|p{10cm}|}
		\caption[Plany rozwojowe aplikacji demonstracyjnej]{
			Plany rozwojowe aplikacji demonstracyjnej
		}
		\label{app:changes_in_future}
		\tabularnewline	
		
		\hline
			\multicolumn{1}{|c|}{\textbf{Moduł}}		&
			\multicolumn{1}{|c|}{\textbf{Opis zmian}}	\tabularnewline
		\hline
		\endfirsthead
		
		\multicolumn{2}{c}
		{{\bfseries \tablename\ \thetable{} -- kontynuacja...}} \tabularnewline
		\hline
			\multicolumn{1}{|c|}{\textbf{Moduł}}		&
			\multicolumn{1}{|c|}{\textbf{Opis zmian}}	\tabularnewline
		\hline
		\endhead
			
		\hline
			\multicolumn{2}{|r|}{{Następna strona...}} \tabularnewline \hline
		\endfoot
		\hline
		\endlastfoot	
		
		\textbf{ComponentBuilder}	&
		Ogólne rozszerzenie i optymalizacja modułu \textbf{ComponentBuilder} dotyczyć będzie:
		\begin{itemize}
			\item wsparcie dla cache'owania raz załadowanych komponentów,
			\item przeniesienie definicji stron do deklaratywnego języka XML.
			Będzie to możliwe po usprawnieniu działania selektywnych konwerterów oraz 
			zwiększy możliwość zmian zgodnie z wymaganiami, bez konieczności zmian w plikach Java,
			\item przeniesienie kodu widoku stron domenowych do łatwiejszego w zarządzaniu oraz utrzymaniu kodu ExtJS,
			\item wprowadzenie automatycznego mechanizmu generującego przekierowania do stron obiektów domenowych,
			\item połączenie kontrolerów odpowiedzialnych za obsługę żądań pochodzących od komponentów typu \textbf{tabele} oraz \textbf{strony domenowe}. Celem tego działania jest ujednolicenie adresów sugerujące, że oba komponenty służą podobnemu celowi.
		\end{itemize}
		\hline		
		\textbf{WebFlow}			&
		Zaprojektowanie biblioteki wspierającej 
		funkcjonalność \textbf{Spring Web Flow} dla biblioteki ExtJS. Decyzja podyktowana jest 
		chęcią zminimalizowania użycia różnorodnych bibliotek JavaScript. Gotowa biblioteka
		byłaby udostępniona na licencji OpenSource.
		\hline
		\textbf{Selektywne konwertery}	&
		Rozszerzenie możliwości selektywnych konwerterów o obiekty inne niż domenowe oraz
		wsparcie dla możliwości definiowana powtarzających się selektorów - kluczy w kontekście
		globalnym, ale unikatowych w kontekście danego obiektu podlegającego konwersji.
		\hline
		\textbf{Kod ogólny}				&
		Usunięcie pozostałości nadmiarowego kodu, który pozostał po uaktualnieniu aplikacji,
		do korzystania z najnowszej wersji (4.0.0) szkieletu aplikacji \textbf{Spring}.
		\hline
		\textbf{Widok}					&
		Wsparcia dla ExtJS - migracja istniejących komponentów warstwy widoku użytkownika do 
		szkieletu aplikacji ExtJS. Prawdopodobnym wyborem będzie wykorzystanie ExtJS w wersji 4.x.x z uwagi
		na lepszą wydajność i większe możliwości biblioteki, w porównaniu do poprzednich wersji.
		\hline
		\textbf{RBuilder}			&
		Zrezygnowania z ciężkiego w użyciu oraz utrzymaniu sposobu generowania raportów w aplikacji poprzez
		bibliotekę \textbf{DynamicJasper}. Przeniesienie bezpośredniego generowania raportów do tabelek oraz
		inwestygacja możliwych technologii do wykorzystania w przypadku eksportowania raportu do formatów takich jak
		HTML, CSV, PDF oraz XLS.		
		\hline
		\textbf{Przewodniki}			&
		Ukończenie wszystkich wymaganych kreatorów służących zarówno do modyfikacji, jak i tworzenia nowych obiektów domenowych.
		\hline
		\textbf{Terminarz spotkań}		&
		Ukończenie prac nad terminarzem spotkań. Obecna funkcjonalność pozwala na tworzenie obiektów, które następnie
		można przeglądać z użyciem kalendarza (podobnego do używanego w programie Outlook). 	
		\hline
		\textbf{Dekodowane numeru VIN}		&
		Rozszerzenie obecnych możliwości dekodera numeru VIN o pobieranie informacji, takich jak:
		\begin{itemize}
			\item wytwórca samochodu,
			\item marka oraz model,
			\item typ nadwozia,
			\item typ paliwa,
			\item typ silnika
		\end{itemize}							
	\end{longtable}
\end{center}