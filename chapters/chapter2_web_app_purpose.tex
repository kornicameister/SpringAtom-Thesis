\chapter{Rola aplikacji internetowych}
\label{chaper:web_app_purpose}

\section{Informatyka jako nauka o przepływie informacji}
	Informatyka jest dziedziną wiedzy, której głównym zadaniem jest badanie przepływu informacji oraz samej informacji. 
	Mówi się, że władzę ma ten, kto ma wiedzę. Nie można zaprzeczyć także, że informacja jest wiedzą. W takim rozumieniu
	informatyka stała się narzędziem, którego używa się obecnie w niemalże wszystkich dziedzinach życia ponieważ usprawnia
	ona przetwarzanie informacji, przechowywanie ich oraz modyfikowanie. 
	W kontekście aplikacji internetowej nie można zapominać o podstawowym celu tej nauki, ponieważ program jest niczym 
	więcej niż zbiorem nic nie znaczących linii kodu, jeśli nie może przetwarzać danych i na ich podstawie generować nowych.
	\todo[inline, color=blue!40]{In progress...}
	
\section{Znaczenie informatyki w świecie biznesu}
	Przedsiębiorstwa inwestowały, inwestują i najprawdopodobniej będą dalej inwestować ogromne sumy pieniędzy na specjalistyczne programy 
	komputerowe całkowicie dopasowane do ich indywidualnych potrzeb, do ich profilu działalności czy tez do procesu produkcyjnego.
	Wynika to z postępującej globalizacji oraz postępu technologicznego. Sam postęp jest tutaj elementem równie ważnym jak ograniczenie 
	granic między państwami. Swoiste \textit{perpetum mobile}, gdzie rozwój wiedzy napędza koniunkturę na coraz bardziej nowoczesne 
	rozwiązanie informatyczne, które z kolei znów napędza rozwój nauk, jest najlepszym przykładem znaczenie informatyki dla biznesu. 
		
	Świadczy to	o sile informatyki jako dziedziny, której obecność w firmie, znacząco podnosi wartość i jakość dostarczanych produktów lub usług. 
	Współcześnie propagowana oraz realizowana koncepcja biznesu opiera się na technologiach informatycznych i bez wątpienia bez nich możliwości
	zaspokajania popytu konsumentów byłyby mocno upośledzone. Wszelkiego rodzaju systemy informatyczne, zintegrowana systemy informatyczne czy
	też nawet programy obsługujące kasy fiskalne są efektem rozwoju nauk informatycznych. Wzajemne przenikanie się obu światów nie jest 
	już zjawiskiem nadzwyczajnym.
	
\section{Przegląd internetowych szkieletów programistycznych}
	\subsection{JBoss Seam Framework}
		\textbf{Seam} był szkieletem aplikacji rozwijanym dla języka Java, zaprojektowanym do wspierania takich obszarów jak: warstwa web z użyciem 
		technologii \textbf{Java Spring Faces}, warstwy logiki biznesowej (\textbf{EJB3}) oraz wykorzystaniem standardu \textbf{JPA} dla warstwy
		danych. Rozwiązuje takie problemy jak:
		\begin{itemize}
			\item tworzenie architektury aplikacji opartego o wzorzec MVC,
			\item związane z naturą aplikacji WWW (sesja, zachowanie stanu itp),
			\item sterowania przepływem oparte o zdarzenia,
			\item zarządzania transakcjami bazodanowymi,
			\item konwersja typów oraz walidacja formularzy,
			\item spójny sposób konfigurowania aplikacji
		\end{itemize}
	
	\subsection{Google Guice}
		\textbf{Google Guice} jest szkieletem aplikacji od \textbf{Google}. Podobnie jak \textbf{Spring} centralnym obszarem wspieranym przezeń jest
		\textit{dependency injection}, które w znaczący sposób poprawia jakość kodu aplikacji poprzez zmniejszenie zależności poszczególnych klas
		między sobą.
	
	\subsection{PicoContainer}
		\textbf{PicoContainer} jest najmniejszym obecnie znanym szkieletem aplikacji wspierającym wzorzec odwróconego sterowania.
		Niemniej z uwagi na swoje rozmiary jest to praktycznie jedyna większa funkcjonalność tego szkieletu. Z uwagi na to jest to dobry wybór jeżeli
		jedynym zadaniem, do którego go potrzebujemy jest zmniejszenie zależności klas w aplikacji. 
		
	\subsection{GWT}
	\subsection{Vaadin}
	\subsection{Struts}