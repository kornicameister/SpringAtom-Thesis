\chapter{Podsumowanie}
\label{chapter:summary}
	Głównymi celami pracy było zaprojektowania oraz przygotowanie aplikacji wspierającej warsztat samochodowy w zakresie
	zarządzania terminarzem wizyt, bazą klientów oraz informacjami o samym warsztacie oraz poznanie szkieletu aplikacji
	Spring jako kompleksowego narzędzia wspierającego tworzenie rozbudowanych aplikacji \textbf{Java EE}. 
	Znaczenia oraz korzyści jakie przynosi korzystania z tego typu aplikacji nie sposób nie zauważyć. Dobrze zaprojektowana
	jest cennym dodatkiem wspomagającym pracę przedsiębiorstwa w zakresie zarówno chwili obecnej jak i analizy danych historycznych.
	Program obejmujący swym zasięgiem globalny aspekt misji danej firmy przy jednoczesnym dostępie do poszczególnych elementów. 
	
	W kontekście aplikacji demonstracyjnej nie udało się zrealizować wszystkich zamierzonych postulatów. Niemniej część z brakujących
	elementów, z uwagi na pracę oraz czas poświęcony na zaprojektowania niewidocznych dla użytkownika elementów, co wcale nie umniejsza
	ich znaczenie, będzie możliwa do szybkiego wprowadzenia do gotowego rozwiązania. Pozostałe braki wynikają głównie z problematyki
	rozwiązania poszczególnych problemów i dla celu aplikacji demonstracyjnej zostały pominięte.
	Mimo że aplikacja przygotowana w ramach projektu inżynierskiego nie jest w pełni funkcjonalna to wiele z jej modułów oraz komponentów
	będzie stanowić o jej sile w momencie gdy braki zostaną uzupełnione. Generyczny kod, który znaleźć można praktycznie w dowolnym miejscu
	oraz nacisk położony na wykorzystanie zestawu bibliotek dodatkowych sprawiają, że późniejsze zmiany czy też usprawnienia nie będą
	trudne. 
	
	Udało się natomiast dobrze poznać podstawowe prawa i reguły rządzące pisaniem rozbudowanego projektu zarówno w sensie ogólnym
	oraz w kontekście framework'a Spring. Rzeczy, które mogą się wydawać trywialne, jak dobry dobór bibliotek, projekt przed implementacją,
	testy jednostkowe, a które zostały zaczerpnięte z zestawu \textbf{best practices} dla Spring pozwoliły zrozumieć na co zwracać szczególną
	uwagę oraz jak ważne mogą stać się małe pomyłki. 