Poniższa tabela przedstawia definicję pojęć technicznych wykorzystanych w pracy dyplomowej. 
\begin{center}
	\begin{longtable}{| p{3cm} | p{13cm} |}
		\caption[Adnotacje Spring opisujące poziom abstrakcji cache]{
			Adnotacje Spring opisujące poziom abstrakcji cache				
		}\tabularnewline	
		
		\hline
			\multicolumn{1}{|c|}{\textbf{Pojęcie}} 		&
			\multicolumn{1}{|c|}{\textbf{Znaczenie}} 		\\
		\hline
		\endfirsthead
		
		\multicolumn{2}{c}
		{{\bfseries \tablename\ \thetable{} -- kontynuacja...}} \\
		\hline
			\multicolumn{1}{|c|}{\textbf{Pojęcie}} 		&
			\multicolumn{1}{|c|}{\textbf{Znaczenie}} 		\\
		\hline
		\endhead
			
		\hline
			\multicolumn{2}{|r|}{{Następna strona...}} 	\\
		\hline
		\endfoot

		\hline
		\endlastfoot	
		% body of the table
		\textbf{Dependency injection}							&
		\label{concept:di}
		Wstrzykiwanie zależności - jest to technika programistyczna, w której dana klasa nie jest odpowiedzialna za tworzenie obiektów od których zależy.
		Zależność jest umieszczana w danym obiekcie, po jego utworzeniu, przez specjalnego zarządcę, posiadającego wiedzę o wszystkich obiektach istniejących w działającej aplikacji.
		Pozwala to jednocześnie na obniżenie ścisłej zależności między dwoma klasami, a także zmniejsza zapotrzebowania na zasoby systemowe. Jest to możliwe, ponieważ, bardzo często,
		obiekty, będące zależnościami dla innych, istnieją w kontekście danego programu, jako \textbf{singletony}. 
		\hline
		\textbf{Singleton}										&
		\textbf{Singleton} jest wzorcem programistycznym, zakładającym istnienie tylko jednego obiektu danej klasy w całej aplikacji. Najczęściej taka funkcjonalność, realizowana jest poprzez
		utworzenie klasy, która nie posiada publicznego konstruktora, a jej obiekt uzyskiwany jest poprzez statyczną metodę, zwracającą obiekt tej klasy. W kontekście szkieletów aplikacji, wspierających
		\textbf{wstrzykiwanie zależności}, wzorzec singletonu realizowany jest na poziomie zarządcy.
		\hline
		\textbf{Zależność przechodnia}							&
		\label{concept:indirect_dependency}
		Taka zależność pojawia się w momencie kiedy aplikacja korzysta z biblioteki, która z kolei
		korzysta z innych. Skutkiem tego jest, że aplikacja staje się zależna od bibliotek od których 
		bezpośrednio nie zależy. Staje się to problematyczne, w momencie kiedy programista
		zaczyna wykorzystywać funkcjonalność zdefiniowaną w nich, nie wiedząc o tym. Usunięcie bezpośredniej zależności,
		skutkować będzie błędami kompilacji.
		\hline
		\textbf{Obiekt domenowy}								&
		\label{concept:domain_object}
		Klasy takich obiektów zdefiniowane są w modelu danych. 
		Innymi słowy dostarczają one informacji opisujących, w sposób abstrakcyjny, 
		rzeczywiste obiekty wykorzystywane w aplikacji.	
		\hline
		\textbf{Repozytorium}									&
		\label{concept:repository}
		Abstrakcyjne pojęcie odnoszące się do klasy obiektów - interfejsów leżących na linii
		baza danych - aplikacja, pośredniczących w operacjach zapisu/odczytu. Implementują ideę
		stojącą za pojęciem CRUD.
		\hline
		\textbf{CRUD}
		\label{concept:crud}									&
		Anglojęzyczny skrót wykorzystywany w programowaniu opisujący 4 podstawowe operacje, wykonywany
		na pewnym źródle danych. 
		\begin{itemize}
			\item create - utworzyć,
			\item read - odczytać,
			\item update - uaktualnić, zmodyfikować, 
			\item delete - usunąć
		\end{itemize}	
		\hline
		\textbf{Cache}											&
		\label{concept:cache}
		Cache jest specjalnym obiektem przeznaczonym do czasowego przechowywania danych
		w pamięci dla zapewnienia szybszego dostępu niż w przypadku odwoływania się do 
		bazy danych, plików lub metod wykonujących kosztowne 
		obliczeniowo operacje.
		\tabularnewline
		\hline
		\textbf{ATP}											&
		\label{concept:atp}
		Pojęcie odnosi się do przetwarzania adnotacji języka Java i wykonywania pewnych operacji
		i/lub generowania wyniku.
		\hline
		\textbf{Adnotacje}										&
		\label{concept:annotation}
		Ideą adnotacji jest dodawanie do kodu źródłowego 
		aplikacji metadanych.
		\hline
		\textbf{AOP}											&
		\label{concept:aop}
		\textbf{Aspect Oriented Programming} jest paradygmatem programistycznym zwiększającym
		skalowalność, zmniejszającym kohezję klas i tym samym eliminującym tak zwane \textit{ścisłe zależności}. 
		Programowanie z jego wykorzystaniem odwołuje się do organizacji kodu w tak zwane \textit{aspekty} realizujące
		pewną logikę. Aspekt swoim zasięgiem jest w stanie objęć więcej niż jeden poziom abstrakcji
		zdefiniowany z użyciem technik programowania obiektowego jak na przykład wzorce strategii 
		czy też fabryk.
		\hline 
		\textbf{Kohezja}										&
		\label{concept:cohesion}
		Kohezja, w odniesieniu do programowania, 
		oznacza stopień w jakim dwie klasy są zależne 
		od siebie.
		\hline 
		\textbf{ORM}										&
		\label{concept:orm}
		Z angielskiego, \textbf{Object/Relational Mapping}, jest to technika programistyczna, w założeniu implementująca
		proces konwertowania danych, które nie są obiektami, jak krotki w bazie danych, na model zaprojektowany, zgodnie
		z wytycznymi programowania obiektowego. W wyniku tego, programista uzyskuje dostęp do wirtualnej bazy danych. 
		\hline 
		\textbf{Getter/Setter}									&
		\label{concept:getter_setter}
		Zwyczajowe pojęcia opisujące metody 
		dostępowe klasy służące do pobierania lub ustawienia 
		wartości pól obiektów tej klasy
		\hline
		\textbf{API}											&
		\label{concept:api}
		Ściśle określony zbiór reguł i metod, dzięki którym program może się komunikować ze sobą lub z innym programem.
		\hline
		\textbf{JMS}											&
		\label{concept:jms}
		\textbf{Java Message Service} to część języka Java, która pozwala dwóm i więcej programom komunikować się
		poprzez jednolity interfejs oraz format wiadomości \cite{java_jms}.
		\hline
		\textbf{JMX}											&
		\label{concept:jmx}
		\textbf{Java Management Extensions} jest technologią, która jest częścią standardowej biblioteki Java.
		Dzięki JMX można kontrolować stan aplikacji i urządzeń oraz dynamicznie wpływać na ich stan \cite{java_jmx}.
		\hline	
		\textbf{Best practice}									&
		\label{concept:best_practices}
		Zalecane i pożądane sposoby realizacji często spotykanych problemów, które można napotkać, podczas projektowania
		aplikacji			
		\hline
		\textbf{DRY}
		\label{concept:dry}										&
		\textbf{Dont Repeat Yourself} - zasada negująca implementacje rozwiązań problemów, które już zostały rozwiązane.
		\hline
		\textbf{Artefakt}										&
		\label{concept:artifact}
		Artefakt, w kontekście programowania obiektowego w Java, jest pojęciem jednocześnie opisującym takie elementy tego
		języka jak klasy, interfejsy oraz paczki (zbiór klas i interfejsów).
		\hline
		\textbf{Predykat}										&
		\label{concept:predicate}
		Predykat, coś co umożliwia ustalenie czegoś. Predykat można rozumieć jako kompozycję wyrażań logicznych lub specjalny 
		rodzaj metody weryfikującej zadany problem pod kątem ustalenia wartości prawda/fałsz.
		\hline
		\textbf{Classpath}										&
		\label{concept:classpath}
		\textbf{Classpath} - ścieżka klas, jest parametrem maszyny wirtualnej Java, który wskazuje na lokalizację folderu w systemie plików, gdzie znajduje 
		się skompilowane klasy Javy.
		\hline
		\textbf{Rewizja}										&
		\label{concept:revision}
		Rewizja jest specjalnym numerem, który jest bezpośrednio związany z historią zmian obiektu domenowego. Każda modyfikacja takiego obiektu oznacza w 
		praktyce utworzenie nowej rewizji. 
		\hline
		\textbf{Lokalizacja}									&
		Lokalizacja, w kontekście dowolnej aplikacji, należy rozumieć jako możliwość programu do wsparcia więcej niż jednego języka interfejsu użytkownika.
		\hline
		\textbf{Przypadek użycia}								&
		Sposób opisu wymagań aplikacji na poziomie interakcji między użytkownikiem aplikacji, a nią samą lub konkretna sytuacja występująca w programie.
	\end{longtable}
	\label{app:ehcache:spring_caches}
\end{center}