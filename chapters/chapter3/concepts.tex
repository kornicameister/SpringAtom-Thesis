\begin{center}
	\begin{longtable}{| p{3cm} | p{13cm} |}
		\caption[Adnotacje Spring opisujące poziom abstrakcji cache]{
			Adnotacje Spring opisujące poziom abstrakcji cache				
		}\tabularnewline	
		
		\hline
			\multicolumn{1}{|c|}{\textbf{Pojęcie}} 		&
			\multicolumn{1}{|c|}{\textbf{Znaczenie}} 		\tabularnewline
		\hline
		\endfirsthead
		
		\multicolumn{2}{c}
		{{\bfseries \tablename\ \thetable{} -- kontynuacja...}} \tabularnewline
		\hline
			\multicolumn{1}{|c|}{\textbf{Pojęcie}} 		&
			\multicolumn{1}{|c|}{\textbf{Znaczenie}} 		\tabularnewline
		\hline
		\endhead
			
		\hline
			\multicolumn{2}{|r|}{{Następna strona...}} \tabularnewline
		\hline
		\endfoot

		\hline
		\endlastfoot	
		
		\textbf{Obiekt domenowy}								&
		\label{concept:domain_object}
		Klasy takich obiektu zdefiniowane są w modelu danych. 
		Innymi słowy dostarczają one definicji informacji 
		opisujących, w sposób abstrakcyjny, 
		rzeczywiste obiekty wykorzystywane w aplikacji.	
		\tabularnewline
		\hline
		\textbf{Repozytorium}									&
		\label{concept:repository}
		Abstrakcyjne pojęcie odnoszące się do klasy obiektów - interfejsów leżących na linii
		baza danych - aplikacja, pośredniczących w operacjach zapisu/odczytu. Implementują ideę
		stojącą za pojęciem CRUD.
		\tabularnewline
		\hline
		\textbf{CRUD}
		\label{concept:crud}									&
		Anglojęzyczny skrót wykorzystywany w programowaniu opisujący 4 podstawowe operacje wykonywany
		na pewnym źródle danych. 
		\begin{itemize}
			\item create - utworzyć,
			\item read - odczytać,
			\item update - uaktualnić, zmodyfikować, 
			\item delete - usunąć
		\end{itemize}	
		\tabularnewline
		\hline
		\textbf{Cache}											&
		\label{concept:cache}
		Cache jest specjalnym obiektem przeznaczonym do czasowego przechowywania danych
		w pamięci dla zapewnienia szybszego dostępu niż w przypadku odwoływania się do 
		bazy danych, plików lub metod wykonujących kosztowne 
		obliczeniowo operacje.
		\tabularnewline
		\hline
		\textbf{ATP}											&
		\label{concept:atp}
		Pojęcie odnosi się do przetwarzania adnotacji języka Java i wykonywania pewnych operacji
		i/lub generowania wyniku.
		\tabularnewline
		\hline
		\textbf{Adnotacje}										&
		\label{concept:annotation}
		Ideą adnotacji jest dodawanie do kodu źródłowego 
		aplikacji metadanych.
		\tabularnewline
		\hline
		\textbf{AOP}											&
		\label{concept:aop}
		\textbf{Aspect Oriented Programming} jest paradygmatem programistycznym zwiększającym
		skalowalność, zmniejszającym kohezję klas i tym samym eliminującym tak zwana \textit{ścisłe zależności}. 
		Programowanie z jego wykorzystaniem odwołuje się do organizacji kodu w tak zwane \textit{aspekty} realizujące
		pewną logikę. Aspekt swoim zasięgiem jest w stanie objęć więcej niż jeden poziom abstrakcji
		zdefiniowanego z użyciem technik programowania obiektowego jak na przykład wzorce strategii 
		czy też fabryk.
		\tabularnewline
		\hline 
		\textbf{Kohezja}										&
		\label{concept:cohesion}
		Kohezja, w odniesieniu do programowania, 
		oznacza stopień w jakim dwie klasy są zależne 
		od siebie.
		\tabularnewline
		\hline 
		\textbf{Getter/Setter}									&
		\label{concept:getter_setter}
		Zwyczajowe pojęcia opisujące metody 
		dostępowe klasy służące do pobierania lub ustawienia 
		wartości pól danej obiektu tej klasy
		\tabularnewline
		\hline
		\textbf{API}											&
		\label{concept:api}
		Ściśle określony zbiór reguł i metod, dzięki którym program może się komunikować ze sobą lub z innym programem.
		\label{concept:api}  
	\end{longtable}
	\label{app:ehcache:spring_caches}
\end{center}